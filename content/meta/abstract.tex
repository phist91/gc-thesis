% !TEX root = ../../thesis.tex
%
\pdfbookmark[0]{Zusammenfassung}{Zusammenfassung}
\chapter*{Zusammenfassung}
\label{sec:abstract}
\vspace*{-10mm}

Die vorliegende Arbeit soll eine Aufarbeitung verschiedener Ansätze für Garbage-Collection-Algorithmen liefern.
Nach einer kurzen Darstellung der zugrunde liegenden Problematik und deren praktische Relevanz sowie den Vor- und Nachteilen einer automatischen Speicherverwaltung gegenüber einer manuellen Speicherverwaltung werden gängige Ansätze vergleichend vorgestellt sowie Einsatz und Eignung in der Praxis beurteilt.
Als Gütekriterien dienen hier beispielsweise Laufzeitbetrachtungen, Speicherbedarf und entstehende Verzögerungen im Programmablauf, die für ausgewählte Ansätze besonders detailliert untersucht werden.

Weiter wird eine Anwendung entworfen, mit der die Arbeitsweise der diskutierten Garbage-Collection-Ansätze visualisiert werden kann.
Dazu gehört eine angemessene Visualisierung eines beschränkten Speicherbereichs, etwa durch eine optische Unterscheidbarkeit belegter Blöcke, sowie der einzelnen Arbeitsphasen, die eine Garbage Collection ausführt.
Dabei sollen auch unterschiedliche Szenarien auswählbar sein, etwa verschiedene Speicherfüllstände und eine variable Anzahl bzw. Größe von Objekten, die im Speicher hinterlegt sind.

\todo[inline]{Am Ende nochmal schauen, ob das wirklich so ist :D}

\vspace*{20mm}

{\usekomafont{chapter}Abstract}\label{sec:abstract-diff} \\
\todo[inline]{Englisch einfügen.}

\blindtext
