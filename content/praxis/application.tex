%!TEX root = ../../thesis.tex
\chapter{Anwendung und Erweiterung}
\label{cha:application}

\todo[inline]{Erweiterungen: Implementation zusätzlicher Algorithmen, zusätzliche Allokatoren, Grafische Ausgabe mit Graph, um Verwaisung direkter beeinflussen und Referenzmanipulationen und -anpassungen sehen zu können, Stepper mit Swing-Timern}

\section{Betrieb des Simulators}
\label{sec:execution}
Der Start der Anwedung erfolgt mithilfe des beigelegten Datenträgers in drei verschiedenen Möglichkeiten:
\begin{enumerate}
	\item Ausführung der JAR-Datei \code{gcsim-1.0-full.jar}.
	\item Ausführung in der Konsole mittels \code{java -jar gcsim-1.0-full.jar}.
	\item Kopieren des Maven-Projekts im Ordner \code{gcsim} und Ausführung eines Build-Zyklus mit \code{mvn package}.
	Anschließend kann der Simulator mit \code{mvn exec:exec} gestartet werden.
\end{enumerate}

Wird zusätzlich der einzige gültige Parameter \code{default} übergeben, werden statt der gespeicherten Einstellungen in der Datei \code{gcsim.config} die Standardeinstellungen verwendet.
Das geschieht auch, wenn keine gültigen gespeicherten Einstellungen vorhanden sind.

Nach dem Start der Anwendung erscheint das Auswahlfenster, in dem grundlegende Einstellungen wie der zu verwendende Garbage-Collection-Algorithmus und die Größe des Heaps festgelegt werden können (siehe Abbildung~\ref{fig:app-start}.
Zudem kann die Größe der Darstellung angepasst werden.
Der Bereich rechts neben den Texteingabefeldern gibt dabei Auskunft über die zu erwartende Größe des Ausgabefensters.
Die Schaltfläche \textit{Einstellungen} ist aktiviert, wenn für den ausgewählten Algorithmus zusätzliche Konfigurationsmöglichkeiten verfügbar sind (siehe Abschnitt~\ref{sec:extension}).

\begin{figure}[h]
	\centering
	\includegraphics[scale=0.5]{img/gui/selection.png}
	\caption[Auswahlfenster des Simulators]{Auswahlfenster des Simulators.}
	\label{fig:app-start}
\end{figure}

Sobald die Auswahl bestätigt wurde, beginnt der Simulationsmodus (siehe Abbildung~\ref{fig:app-simulation}).
Mithilfe der Eingabekomponenten im oberen Bereich des Kontrollfensters können nun neue Objekte zum Heap hinzugefügt werden.
Die beiden oberen Textfelder \textit{untere Grenze} und \textit{obere Grenze} geben dabei die Größenordnung an, in der sich neu erzeugte Objekte befinden.
Das Textfeld \textit{Verzweigungsgrad} gibt die Wahrscheinlichkeit an, mit der jedes bereits existierende Objekt eine Referenz auf ein neu erstelltes Objekt erhält, während \textit{Anteil Basisobjekte} bestimmt, mit welcher Wahrscheinlichkeit ein erzeugtes Objekt ein Basisobjekt ist.
Die Schaltfläche \textit{Objekt erstellen} erzeugt ein einzelnes Objekt mit den spezifizierten Eigenschaften, während ein Klick auf \textit{Heap füllen} solange Objekte erzeugt, bis ein Allokationsversuch fehlschlägt.
Mittels der Schaltfläche \textit{Heap leeren} kann der Heap zurückgesetzt werden.
Dies ist etwa notwendig, wenn ein Kollektionszyklus unterbrochen wird und sich Objekte mit verschiedenen Markierungen im Heap befinden.
Die Schaltfläche \textit{Heap ausblenden} verbirgt die grafische Ausgabe des Heaps bzw. blendet sie wieder ein.
Im unteren Bereich des Kontrollfensters lässt sich mit dem \textit{Verwaisungsgrad} die Wahrscheinlichkeit einstellen, mit der eine Referenz entfernt wird (siehe auch Abschnitt~\ref{sub:controller}).
Im Textfeld \textit{Animationsgeschwindigkeit} kann der zeitliche Abstand zweier Animationsschritte festgelegt werden (siehe auch Abschnitt~\ref{sec:gui}).
Ein Klick auf \textit{GC ausführen} löst den Garbage-Collection-Algorithmus aus.

\begin{figure}[h]
	\centering
	\begin{subfigure}[t]{0.6\textwidth}
		\centering
		\includegraphics[scale=0.5]{img/gui/simulation-control.png}
		\caption{Kontrollfenster}
	\end{subfigure}~\hspace{0.25cm}~
	\begin{subfigure}[t]{0.35\textwidth}
		\centering
		\includegraphics[scale=0.5]{img/gui/simulation-canvas.png}
		\caption{Darstellung des Heaps}
	\end{subfigure}
	\caption[Simulationsmodus]{Simulationsmodus.}
	\label{fig:app-simulation}
\end{figure}



\section{Erweiterungsmöglichkeiten}
\label{sec:extension}