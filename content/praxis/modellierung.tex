%!TEX root = ../../thesis.tex
\chapter{Modellierung und Implementation}
\label{cha:modelling}

\todo[inline]{Intro}

\section{Spezifikation der Anforderungen}
\label{sec:requirements}
Der zu entwerfende Simulator soll die Ausführung ausgewählter Garbage-Collection-Algorithmen demonstrieren, sodass die einzelnen Arbeitsschritte eines Algorithmus klar erkennbar sind.
Dazu soll das in Kapitel~\ref{cha:intro} eingeführte Speichermodell realisiert werden.
Eine grafische Ausgabe visualisiert dabei den Heap als Blockgrafik: Belegte Teile des Heaps sollen von freien optisch unterschieden werden können, zudem sollen Lage und Größe der einzelnen Objekte erkennbar sein.
Die Anwenderin soll darüber hinaus die Möglichkeit haben, den Simulator zu konfigurieren:
Neben einer obligatorischen Auswahl des verwendeten Garbage-Collection-Algorithmus soll auch die Größe des Heaps sowie die Größe der erzeugten Heapobjekte einstellbar sein.
Weitere Anpassungsmöglichkeiten sind der Verzweigungsgrad der Objekte untereinander sowie die Verwaisungswahrscheinlichkeit, sodass verschiedene Objektkonstellationen betrachtet werden können.
Zuletzt soll durch eine Anpassbarkeit der Animationsgeschwindigkeit auch die Visualisierung anpassbar sein.

Aus Sicht der Softwareentwicklung ergeben sich zusätzliche Anforderungen, die für die Umsetzung der Anwendung relevant sind:
Der Simulator soll als frei verwendbare Software im Rahmen der Lehre eingesetzt werden können und beliebig anpassbar und erweiterbar sein, etwa indem Entwicklerinnen weitere Garbage-Collection-Algorithmen ergänzen.
Dies impliziert die Verwendung verschiedener Konzepte der Objektorientierung, unter anderem der Modularisierung in funktional unterscheidbare Pakete und Klassen, der Polymorphie und der generischen Programmierung.
Darüber hinaus soll die Anwendung plattformunabhängig entwickelt werden, um sie möglichst vielen Anwenderinnen zugänglich zu machen.