%!TEX root = ../../thesis.tex
\chapter{Qualitativer Vergleich}
\label{cha:comparision}

In den vorigen Kapiteln wurde eine Reihe sowohl grundlegender als auch elaborierter Algorithmen diskutiert, deren Vor- und Nachteile erläutert sowie Einsatzfälle genannt, in denen sie potenziell ihre Stärken ausspielen können.
Zum Abschluss des ersten Teils sollen daher nochmals die Algorithmen bezüglich ausgewählter qualitativer Kriterien vergleichend gegenübergestellt werden.
Diese Kriterien können je nach Anwendungsfall für eine erste Auswahl potenziell infrage kommender Garbage-Collection-Algorithmen herangezogen werden.


% % % % % % % % % % % % % % % % % % % % % % % % % % % % % % % % % % 
 % % % % % % % % % % % % % % % % % % % % % % % % % % % % % % % % % % 
% % % % % % % % % % % % % % % % % % % % % % % % % % % % % % % % % % 


\minisec{Durchsatz oder Responsivität?}
Der Durchsatz ist ein Maß für die Performanz einer Software und beschreibt die Geschwindigkeit, mit der diese Aufgaben erledigt.
Ein hoher Durchsatz verlangt, dass der Kollektor das eigentliche Programm möglichst wenig ausbremst und dem Mutator möglichst viel Prozessorzeit überlässt.
Die Responsivität beschreibt hingegen die Antwortzeit, mit der eine Anwendung auf eingehende Anfragen reagiert.
Wir haben gesehen, dass markierende Algorithmen häufig die Arbeit des Mutators unterbrechen, um fehlerhafte Markierungen zu vermeiden.
In dieser Zeit können eingehende Anfragen nicht verarbeitet werden, da sie durch den Kollektor blockiert werden -- die Responsivität sinkt.
Referenzzählende Algorithmen verteilen ihre Arbeit hingegen auf die einzelnen Schreibzugriffe des Mutators, sodass hier Performanceeinbußen zu erwarten sind, während die Responsivität weitgehend erhalten bleibt.
In ihrer naivsten Form scheint daher die Wahl klar zu sein:
Strebt man eine Optimierung des Durchsatzes an, fällt die Entscheidung auf den Mark-Sweep-Algorithmus, da dieser außerhalb eines Kollektionszyklus keine zusätzliche Arbeit verursacht.
Soll hingegen die Responsivität optimiert werden, führt kein Weg an der Referenzzählung vorbei, da sie keine längerfristigen Unterbrechungen verursacht.
Allerdings war auch zu sehen, dass die fortgeschritteneren Varianten der beiden Algorithmen diese Klarheit relativieren.
So kann die Drei-Farben-Abstraktion die Responsivität erhöhen, indem teilweise die Nebenläufigkeit von Mutator und Kollektor ermöglicht wird.
Hierzu sind jedoch -- wie bei der Referenzzählung -- Schreibbarrieren nötig, die sich negativ auf den Durchsatz auswirken können.
Soll die Referenzzählung auch zyklische Strukturen zuverlässig freigeben oder aus Durchsatzgründen auf einige Schreibbarrieren verzichten, sind wiederum separate Bereinigungsphasen notwendig, die den Mutator anhalten und die Referenzzähler korrigieren.


% % % % % % % % % % % % % % % % % % % % % % % % % % % % % % % % % % 
 % % % % % % % % % % % % % % % % % % % % % % % % % % % % % % % % % % 
% % % % % % % % % % % % % % % % % % % % % % % % % % % % % % % % % % 


\minisec{Speicherbedarf}
Da eine Garbage Collection häufig ausgelöst wird, wenn der freie Heapspeicher zur Neige geht, ist der Speicherbedarf der verschiedenen Konzepte ein weiterer Aspekt, der bei der Wahl des passenden Algorithmus beachtet werden sollte.
Alle vorgestellten Algorithmen benötigen zusätzlichen Speicher pro Objekt, sei es, um Markierungsinformationen oder Referenzzähler innerhalb oder außerhalb von Objekt"-headern zu speichern.
Einzelne Bits, die von markierenden Algorithmen verwendet werden, können gegebenenfalls in bereits vorhandene Header integriert werden, Referenzzähler können jedoch prinzipiell unbeschränkt groß werden.
Für diese muss dann während der gesamten Laufzeit der Anwendung Speicher bereitgehalten werden.
Der dadurch verursachte Overhead hängt vor allem von der Anzahl der Objekte und deren Größe ab.
Auch die Aufteilung des Heaps in Blöcke macht die Speicherung zusätzlicher Informationen zur Verwaltung selbiger notwendig.
Die Algorithmen selbst benötigen meist Hilfsdatenstrukturen wie Warteschlangen, um Adressen noch zu untersuchender Objekte zwischenzuspeichern sowie Änderungen von Speicheradressen nachzuverfolgen.
Entsprechend viel Speicher muss ebenfalls reserviert werden, damit die Algorithmen ordnungsgemäß arbeiten können.
Als besonders speicherhungrig erweisen sich hier der Compressor-Algorithmus (siehe Tabelle~\ref{tab:compressor} auf Seite~\pageref{tab:compressor}) sowie das Halbraumverfahren aus Abschnitt~\ref{sec:copying}, welches die Heapgröße praktisch halbiert.
Für Anwendungen, in denen wenig Speicher zur Verfügung steht, kann dies ein Ausschlusskriterium sein, da häufige Ausführungen der Garbage Collection den Mutator verdrängen könnten.


% % % % % % % % % % % % % % % % % % % % % % % % % % % % % % % % % % 
 % % % % % % % % % % % % % % % % % % % % % % % % % % % % % % % % % % 
% % % % % % % % % % % % % % % % % % % % % % % % % % % % % % % % % % 


\minisec{Kopieren oder nicht kopieren?}
Ein weiteres Kriterium bei der Suche eines geeigneten Garbage"=Collection"=Algorithmus ist, ob der Algorithmus den genutzten Heapspeicher kompaktiert oder nicht.
Bildet der freie Speicher einen möglichst großen zusammenhängenden Bereich, können aufeinanderfolgende Speicheranforderungen schnell erfüllt werden, da keine \textit{passende Lücke} für die angeforderte Speichermenge gesucht werden muss.
Allerdings sind kopierende Algorithmen vor allem für große Heaps tendenziell langsamer, da im schlimmsten Fall alle erreichbaren Objekte verschoben werden, sämtliche Referenzen auf verschobene Objekte angepasst werden müssen und der Mutator in dieser Zeit angehalten wird.
Zumal besteht die Gefahr, dass der Kollektor \textit{gegen den Allokator} arbeitet, da kopierende Algorithmen die Reihenfolge der Objekte verändern könnte, die zuvor vom Allokator zur Ausnutzung räumlicher Lokalität optimiert wurde.
Insofern ist es wichtig, den Kollektor auch auf die Arbeitsweise des Allokators abzustimmen.


% % % % % % % % % % % % % % % % % % % % % % % % % % % % % % % % % % 
 % % % % % % % % % % % % % % % % % % % % % % % % % % % % % % % % % % 
% % % % % % % % % % % % % % % % % % % % % % % % % % % % % % % % % % 


\minisec{Quantitative Resultate}
Neben dieser qualitativen Gegenüberstellung ist es durchaus möglich, auch die quantitativen Unterschiede der verschiedenen Konzepte zu messen.
Vor allem neuere Algorithmen werden dazu für die \textit{Jikes Research Virtual Machine}\footnote{\url{https://www.jikesrvm.org/}} implementiert oder Benchmark-Frameworks wie \textit{SPECjvm}\footnote{\url{https://www.spec.org/jvm2008/}} ausgesetzt, um Performanceunterschiede messbar zu machen und dabei verschiedene Ausprägungen von Objekten und Objektkonstellationen zu betrachten (siehe etwa \cite{blackburn2003}, \cite{blackburn2004}, \cite{kermany2006}, \cite{levanoni2006}, \cite{garner2007}).
Diese belegen durchaus, dass zum einen deutliche Performanceunterschiede zwischen verschiedenen Ansätzen in gewissen Anwendungsfällen existieren und zum anderen vermutlich kein Algorithmus existiert, der in jeder Situation perfekt ist (vgl. \cite{fitzgerald2000}).
Dennoch sollte beachtet werden, dass derartige Untersuchungen unter gewissen \textit{Laborbedingungen} stattfinden, da sie Anwendungsfälle lediglich simulieren.
Somit lassen sich die Resultate nur eingeschränkt auf die Praxis übertragen, weswegen hier abschließend nur einer Empfehlung zugestimmt werden kann: \enquote{Know your application} \cite[S. 80]{handbook}.
