%!TEX root = ../../thesis.tex
\chapter{Qualitativer Vergleich}
\label{cha:comparision}

In den vorigen Kapiteln wurde eine Reihe sowohl grundlegender als auch elaborierter Algorithmen diskutiert, deren Vor- und Nachteile erläutert sowie Einsatzfälle genannt, in denen sie potenziell ihre Stärken ausspielen können.
Zum Abschluss des ersten Teils sollen daher nochmals die Algorithmen vergleichend gegenübergestellt werden.
% Dieser Vergleich ist an Kriterien orientiert, deren Optimierung 


% % % % % % % % % % % % % % % % % % % % % % % % % % % % % 


\subsubsection*{Durchsatz oder Responsivität?}
Der Durchsatz ist ein Maß für die Performanz einer Software und beschreibt die Geschwindigkeit, mit der diese Aufgaben erledigt.
Ein hoher Durchsatz verlangt, dass der Kollektor das eigentliche Programm möglichst wenig ausbremst und dem Mutator möglichst viel Prozessorzeit überlässt.
Die Responsivität beschreibt hingegen die Antwortzeit, mit der eine Anwendung auf eingehende Anfragen reagiert.
Wir haben gesehen, dass markierende Algorithmen häufig die Arbeit des Mutators unterbrechen, um fehlerhafte Markierungen zu vermeiden.
In dieser Zeit können eingehende Anfragen nicht verarbeitet werden, da sie durch den Kollektor blockiert werden -- die Responsivität sinkt.
Referenzzählende Algorithmen verteilen ihre Arbeit hingegen auf die einzelnen Schreibzugriffe des Mutators, sodass hier Performanceeinbußen zu erwarten sind, während die Responsivität weitgehend erhalten bleibt.
In ihrer naivsten Form scheint daher die Wahl klar zu sein:
Strebt man eine Optimierung des Durchsatzes an, fällt die Entscheidung auf den Mark-Sweep-Algorithmus, da dieser außerhalb eines Kollektionszyklus keine zusätzliche Arbeit verursacht.
Soll hingegen die Responsivität optimiert werden, führt kein Weg an der Referenzzählung vorbei, da sie keine Unterbrechungen verursacht.
Allerdings haben wir auch gesehen, dass die fortgeschritteneren Varianten der beiden Algorithmen diese Klarheit relativieren.
So kann die Drei-Farben-Abstraktion die Responsivität erhöhen, indem teilweise die Nebenläufigkeit von Mutator und Kollektor ermöglicht wird.
Allerdings sind dazu -- wie bei der Referenzzählung -- Schreibbarrieren nötig, die sich negativ auf den Durchsatz auswirken können.
Soll die Referenzzählung auch zyklische Strukturen zuverlässig freigeben oder aus Durchsatzgründen auf einige Schreibbarrieren verzichten, sind wiederum separate Bereinigungsphasen notwendig, die den Mutator anhalten und die Referenzzähler korrigieren.

