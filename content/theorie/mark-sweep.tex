%!TEX root = ../../thesis.tex
\chapter{Mark-Sweep-Algorithmen}
\label{cha:mark-sweep}
Dieses Kapitel beginnt mit einer Vorstellung des ersten Garbage-Collection-Algorithmus, der auf \textsc{McCarthy} zurückgeht \cite[191--193]{mccarthy1960}.
Im Rahmen eines im Jahr 1960 veröffentlichten Artikels über die Berechnung rekursiver Funktionen auf dem \textit{IBM 704} mithilfe des \textit{LISP Programming Systems} erläutert \textsc{McCarthy} die Speicherung von Daten in einer Listenstruktur.
Diese besteht aus Paaren, deren erster Eintrag \texttt{car} die zu speichernde Information enthält\footnote{Genauer: Die Adresse des Registers, in dem die zu speichernde Information gelagert wird.}, während im zweiten Eintrag \texttt{cdr} die Registeradresse des nachfolgenden Paares zu finden ist.
Diese Struktur hat den Vorteil, dass ein in mehreren Listen vorkommendes Datum nur ein einziges Register belegt.
Register, die aktuell nicht zur Speicherung von Daten genutzt werden, befinden sich in einer \textit{free storage list}.
Bei der Anforderung von Speicher für ein zu speicherndes Datum werden Register aus dieser Liste entfernt.
Durch die Manipulation der Registeradressen können Paare verwaisen, was zu Speicherlecks führt.
Zur Auflösung dieser Problematik bietet LISP als erste Programmiersprache ihrer Zeit eine automatische Speicherverwaltung, die von \textsc{McCarthy} wie folgt grob umschrieben wird:
Im Falle von Speicherknappheit wird -- ausgehend von einer Menge von Basisregistern -- ermittelt, welche Register über eine Folge von \texttt{cdr}-Einträgen erreichbar sind.
Nicht erreichbare Register enthalten überschreibbare Inhalte, sodass diese zurück in die \textit{free storage list} eingefügt werden können und wieder als freie Speicherplätze zu Verfügung stehen.
Diese zweischrittige Vorgehensweise -- das Erkennen nicht mehr benötigter Speicherbereiche und die anschließende Freigabe eben jener -- bildet die Grundlage des \textit{Mark-Sweep-Algorithmus}.

\begin{figure}[h]
	\centering
	\includestandalone{img/tikz/ch2-lisp}
	\caption[Visualisierung einer LISP-Liste]{Visualisierung der LISP-Liste \texttt{a = (13,42,7,42)} als \textit{Box-and-Pointer}-Diagramm (vgl. \cite[Kapitel 3.3]{sicp}).}
	\label{fig:lisp-list}
\end{figure}


% % % % % % % % % % % % % % % % % % % % % % % % % % % % % % % % % % 
 % % % % % % % % % % % % % % % % % % % % % % % % % % % % % % % % % % 
% % % % % % % % % % % % % % % % % % % % % % % % % % % % % % % % % % 


\section{Naiver Mark-Sweep-Algorithmus}
\label{sec:naive-mark-sweep}
Der naive Mark-Sweep-Algorithmus arbeitet in zwei Schritten:
Zunächst wird bestimmt, welche Objekte im Speicher unerreichbar sind, weil sie von keinem anderen erreichbaren Objekt referenziert werden.
Diese Objekte können gefahrlos freigegeben werden, da auf ihre Informationen nicht mehr zugegriffen werden kann.
Der zweite Schritt besteht aus einer Traversierung des gesamten Heaps.
Dabei werden alle existierenden Objekte besucht und diejenigen freigegeben, die im ersten Schritt als unerreichbar identifiziert werden konnten.
Die entsprechenden Speicherbereiche stehen anschließend wieder für neue Objekte zu Verfügung.

\begin{algorithm}
\begin{algorithmic}[1]
	\State \MethodHead{collect}():
	\State \quad \Method{markStart}()
	\State \quad \Method{sweep}()
	\Statex
	\State \MethodHead{markStart}():
	\State \quad \Var{toDo} $\gets$ $\emptyset$				\Comment{Noch abzuarbeitende Objekte}
	\State \quad \FOREACH \Var{obj} $\in \Roots$		\Comment{Beginne mit Basisobjekten}
	\State \quad \quad \IF \NOT \Method{isMarked}(\Var{obj})
	\State \quad \quad \quad \Method{setMarked}(\Var{obj})	\Comment{Objekt als erreichbar markieren}
	\State \quad \quad \quad \Method{add}(\Var{toDo}, \Var{obj})	
	\State \quad \quad \quad \Method{mark}()			\Comment{Abarbeitung starten}
	\Statex
	\State \MethodHead{mark}():
	\State \quad \WHILE \Var{toDo} $\neq \emptyset$
	\State \quad \quad \Var{obj} $\gets$ \Method{remove}(\Var{toDo})			\Comment{Hole nächstes Objekt}
	\State \quad \quad \FOREACH \Var{ref} $\in \Pointers(\Var{obj})$	\Comment{Hole nächste Referenz auf Objekt}
	\State \quad \quad \quad \IF \NOT \Method{isMarked}(\Var{*ref})
	\State \quad \quad \quad \quad \Method{setMarked}(\Var{*ref})	
	\State \quad \quad \quad \quad \Method{add}(\Var{toDo}, \Var{*ref})
\end{algorithmic}
\caption[Naives Mark and Sweep -- Markierung]{Naives Mark and Sweep -- Markierung (vgl. \cite[Kap. 2.2]{jones-lins})}
\label{algo:naive-mark}
\end{algorithm}

Die Markierungsphase (engl. \textit{mark}) startet mit Auslösung der Garbage Collection durch den Allokator und funktioniert wie folgt:
Ein Objekt zu markieren bedeutet, es als erreichbar zu kennzeichnen, indem in seinem Header ein entsprechender Wert -- etwa ein Bit -- gesetzt wird.
Zunächst wird mittels der Prozedur \Method{markStart} eine Menge \Var{toDo} erzeugt, die diejenigen Objekte enthält, die bereits als erreichbar erkannt, aber selbst noch nicht verarbeitet wurden (Zeile 5 in Algorithmus~\ref{algo:naive-mark}).
In diese werden alle bislang unmarkierten Basisobjekte der Menge \Roots eingefügt und markiert, da sie in jedem Fall erreichbar sind (Zeile 6 bis 9).
Ist ein Basisobjekt bereits markiert worden, so wurde es schon entdeckt -- etwa, weil es durch ein zuvor abgearbeitetes Objekt referenziert wird.
Daraus folgt, dass es ebenfalls bereits abgearbeitet wurde oder sich noch in der Menge \Var{toDo} befindet.
In beiden Fällen muss es folglich nicht erneut zu \Var{toDo} hinzugefügt werden.

Bereits nach dem Hinzufügen des ersten Basisobjekts wird die Prozedur \Method{mark} aufgerufen, welche die \Var{toDo}-Menge abarbeitet.
Für jedes Objekt in \Var{toDo} werden diejenigen Felder betrachtet, die eine Referenz auf ein Objekt enthalten (Zeile 13 und 14).
Wenn dieses Objekt noch nicht markiert wurde, wird es in diesem Augenblick zum ersten Mal entdeckt.
Da es somit erreichbar ist, kann es markiert und zu \Var{toDo} hinzugefügt werden, um zu einem späteren Zeitpunkt abgearbeitet zu werden (Zeile 15 und 16).
Verweist die Referenz hingegen auf ein Objekt, das bereits markiert wurde, wurde dieses schon zuvor entdeckt.
Auch hier ist ein erneutes Hinzufügen zu \Var{toDo} überflüssig.
Sobald \Var{toDo} leer ist, erfolgt die Rückkehr zur Prozedur \Method{markStart}, sodass gegebenenfalls das nächste Basisobjekt abgearbeitet wird.
Abbildung~\ref{fig:mark-example} zeigt die Arbeitsweise des Algorithmus an einem Beispiel.

\begin{figure}[H] \newcommand{\markexscale}{0.85} \newcommand{\markexspace}{0.7cm}
	\centering
	\begin{subfigure}[t]{0.45\textwidth}
		\centering
		\includestandalone[scale=\markexscale]{img/tikz/ch2-mark1}
	\end{subfigure}~\hspace{\markexspace}~
	\begin{subfigure}[t]{0.45\textwidth}
		\centering
		\includestandalone[scale=\markexscale]{img/tikz/ch2-mark2}
	\end{subfigure}\\[1cm]
	\begin{subfigure}[t]{0.45\textwidth}
		\centering
		\includestandalone[scale=\markexscale]{img/tikz/ch2-mark3}
	\end{subfigure}~\hspace{\markexspace}~
	\begin{subfigure}[t]{0.45\textwidth}
		\centering
		\includestandalone[scale=\markexscale]{img/tikz/ch2-mark4}
	\end{subfigure}\\[1cm]
	\begin{subfigure}[t]{0.45\textwidth}
		\centering
		\includestandalone[scale=\markexscale]{img/tikz/ch2-mark5}
	\end{subfigure}~\hspace{\markexspace}~
	\begin{subfigure}[t]{0.45\textwidth}
		\centering
		\includestandalone[scale=\markexscale]{img/tikz/ch2-mark6}
	\end{subfigure}\\[0.5cm]
	\caption[Beispielhafte Ausführung der Markierungsphase]{Beispielhafte Ausführung von Algorithmus~\ref{algo:naive-mark}.}
	\label{fig:mark-example}
\end{figure}

Es ist wesentlich, dass Objekte bereits markiert werden, wenn sie der Menge \Var{toDo} hinzugefügt werden, und nicht etwa, nachdem sie abgearbeitet wurden (Zeile 8 und 9 bzw. 16 und 17).
Andernfalls besteht bei zyklischen Referenzen die Gefahr einer Endlosschleife, da unmarkierte Objekte mehrfach hinzugefügt würden.
Präziser lässt sich festhalten, dass \Var{toDo} zu jedem Zeitpunkt ausschließlich bereits markierte Objekte enthält.
Da keine Objekte hinzugefügt werden, die bereits markiert wurden (Zeile 7 und 15), wird kein Objekt mehrfach verarbeitet.
Da zudem mit jeder Iteration der \WHILE-Schleife mindestens ein Objekt aus \Var{toDo} entfernt wird (Zeile 13), die Anzahl aller Objekte endlich ist und vorausgesetzt, dass während der Ausführung des Kollektors keine neuen Objekte entstehen, wird sowohl die \WHILE-Schleife, als auch die \FOREACH-Schleife nach endlich vielen Schritten terminieren.

\begin{algorithm}
\begin{algorithmic}[1]
	\State \MethodHead{sweep}():
	\State \quad \Var{pos} $\gets$ \Method{nextObject}(\Var{HEAP\_START})
	\State \quad \WHILE \Var{pos} $\neq$ \Null
	\State \quad \quad \IF \Method{isMarked}(\Var{*pos})
	\State \quad \quad \quad \Method{unsetMarked}(\Var{*pos})
	\State \quad \quad \ELSE \Method{free}(\Var{pos})
	\State \quad \quad \Var{pos} $\gets$ \Method{nextObject}(\Var{pos})
\end{algorithmic}
\caption[Naives Mark and Sweep -- Bereinigung]{Naives Mark and Sweep -- Bereinigung (vgl. \cite[Kap. 2.2]{jones-lins})}
\label{algo:naive-sweep}
\end{algorithm}

Die Bereinigungsphase (engl. \textit{sweep}) beginnt unmittelbar nach der Markierungsphase durch Aufruf der Prozedur \Method{sweep}.
Die Variable \Var{pos} wird mit der Speicheradresse initialisiert, an der sich das erste Objekt im Heap befindet.
Hier wird davon ausgegangen, dass eine Prozedur \Method{nextObject} des Allokators zu Verfügung steht, die anhand einer übergebenen Speicheradresse die Adresse des nachfolgenden Objektes oder \Null zurückgibt, wenn dieses nicht existiert.
Dadurch wird der Heap linear traversiert; nicht markierte Objekte werden freigegeben, während die Markierung erreichbarer Objekte zurückgesetzt wird.

Wir halten zunächst fest, dass der Mark-Sweep-Algorithmus in seiner Gänze terminiert und korrekt ist, sofern während der Garbage Collection das laufende Programm angehalten wird:

\begin{mybox}
\begin{satz}[Korrektheit des naiven Mark-Sweep-Algorithmus]
	\label{satz:mark-sweep-correctness}
	Der Mark-Sweep-Algorithmus terminiert und ist korrekt, wenn der Mutator während der Arbeit des Kollektors angehalten wird.
\end{satz}
\end{mybox}

\begin{proof}
	Wie oben erläutert, terminiert die Markierungsphase in jedem Fall, da bei angehaltenem Mutator keine neuen Objekte erstellt werden.
	Gleiches gilt für die Bereinigungsphase, in der alle Objekte des Heaps in endlicher Zeit besucht werden.
	Somit terminiert der gesamte Algorithmus.
	
	Weiter werden lediglich nicht markierte Objekte freigegeben (Zeile~4 bis~6 in Algorithmus~\ref{algo:naive-sweep}).
	Bleibt ein Objekt \Var{obj} unmarkiert, so ist \Var{obj} kein Basisobjekt, da \Method{markStart} alle Basisobjekte markiert.
	Somit wird kein Basisobjekt freigegeben.
	Wir zeigen, dass es nach der Markierungsphase zudem keine zwei Objekte $\Var{a}$, $\Var{b}$ mit $\Var{a} \rightarrow \Var{b}$ gibt, sodass $\Var{a}$ markiert und $\Var{b}$ unmarkiert ist:
	Da $\Var{a}$ markiert ist, wurde $\Var{a}$ auch der Menge $\Var{toDo}$ hinzugefügt (Zeile 8f bzw. 16f in Algorithmus~\ref{algo:naive-mark}).
	Entsprechend gab es eine Iteration der \WHILE-Schleife mit $\Var{obj} = \Var{a}$.
	Gilt nun \Var{a} $\rightarrow$ \Var{b}, so ist $\Var{\&b} \in \Pointers(\Var{a})$.
	Folglich wird in einer Iteration der \FOREACH-Schleife mit \Var{ref} $=$ \Var{\&b} auch \Var{*(\&b)} $=$ \Var{b} markiert, falls \Var{b} nicht schon zuvor markiert wurde.
	Es existiert somit keine Referenz von einem erreichbaren auf ein unmarkiertes Objekt, weswegen keine erreichbaren Objekte freigegeben werden.
\end{proof}

Die Bedingung, dass der Mutator während des Markierens pausiert wird, ist tatsächlich notwendig, um zu vermeiden, dass fälschlicherweise erreichbaren Objekte entfernt werden, wie folgendes Beispiel zeigt (vgl. \cite[969]{dijkstra1978}):
Betrachtet man etwa die Situation, dass zwei Basisobjekte $\Var{a}$ und $\Var{b}$ alternierend auf ein Objekt $\Var{c}$ verweisen, das ausschließlich über $\Var{a}$ oder $\Var{b}$ erreichbar ist.
Während der Kollektor aktiv ist, führe der Mutator folgenden Code aus:

\begin{center}
\begin{minipage}{0.3\textwidth}
	\centering
	\begin{algorithmic}[1]
		\State \Var{b.ref} $\gets$ \Var{\&c}
		\State \Var{a.ref} $\gets$ \Null
		\State \Var{a.ref} $\gets$ \Var{\&c}
		\State \Var{b.ref} $\gets$ \Null
	\end{algorithmic}
\end{minipage}~
\begin{minipage}{0.3\textwidth}
	\centering
	\includestandalone{img/tikz/ch2-concurrent1}
\end{minipage}~
\begin{minipage}{0.3\textwidth}
	\centering
	\includestandalone{img/tikz/ch2-concurrent2}
\end{minipage}
\end{center}

Es könnte passieren, dass der Kollektor gerade Objekt $\Var{a}$ abarbeitet, unmittelbar nachdem Zeile~2 ausgeführt wurde.
Es wird dann keine Referenz auf Objekt $\Var{c}$ vorgefunden.
Wenn der Kollektor nun Objekt $\Var{b}$ betrachtet, nachdem bereits Zeile~4 abgearbeitet wurde, wird Objekt $\Var{c}$ weiterhin nicht entdeckt.
Insgesamt wird Objekt $\Var{c}$ somit nicht markiert, obwohl es erreichbar ist.
In der Folge würde $\Var{c}$ irrtümlich freigegeben werden, sodass ein hängender Zeiger entsteht oder sogar Datenverlust verursacht wird -- der Algorithmus arbeitet also nicht korrekt.

Situationen, in denen die nebenläufige Ausführung von Programmsegmenten zu unvorhersehbarem Verhalten führt, werden als \textit{race conditions} bezeichnet.
Algorithmen, die zur Vermeidung von \textit{race conditions} zwischen Kollektor und Mutator die Arbeit des letzteren unterbrechen, werden auch als \textit{Stop-the-World-Algorithmen} (vgl. \cite[S. 2]{levanoni2006}) bezeichnet.


% % % % % % % % % % % % % % % % % % % % % % % % % % % % % % % % % % 
 % % % % % % % % % % % % % % % % % % % % % % % % % % % % % % % % % % 
% % % % % % % % % % % % % % % % % % % % % % % % % % % % % % % % % % 


\section{Drei-Farben-Abstraktion}
\label{sec:tricolor}
Da das Anhalten des Mutators während eines gesamten Garbage-Collection-Zyklus zu vergleichsweise großen Verzögerungen führt, ist eine Optimierung der Markierungsphase wünschenswert.
Zielführend ist, Kollektor und Mutator möglichst häufig eine nebenläufige Ausführung zu ermöglichen, ohne dabei die Korrektheit der Garbage Collection zu gefährden.
Wir haben gesehen, dass die Manipulation von Referenzen während der Markierungsphase dazu führt, dass Verweise von markierten auf unmarkierte Objekte entstehen können, sodass erreichbare Objekte unmarkiert bleiben.
Um dies zu vermeiden, könnte man als ersten Ansatz auf die Idee kommen, beim Schreiben einer neuen Referenz in ein Feld eines Objekts das Ziel dieser Referenz sofort zu markieren.
Dies ist jedoch nur scheinbar eine Lösung:
Da das Ziel ebenfalls Referenzen auf unmarkierte Objekte bereithalten könnte, entstehen dadurch möglicherweise neue Referenzen von markierten auf unmarkierte Objekte.
Von \textsc{Dijkstra} et al. stammt ein Ansatz, der diese Idee aufgreift und um ein \textit{Zwischenstadium} erweitert \cite[S. 969f]{dijkstra1978}.
Diese ermöglicht es, markierte Objekte, die bereits komplett abgearbeitet wurden, von solchen zu unterscheiden, die bislang lediglich entdeckt wurden.
Dazu werden Objekte mit drei verschiedenen Farben markiert:

\begin{description}
	\item[weiß:] Das Objekt wurde bislang nicht als erreichbar identifiziert.
		Bleibt es nach Ende der Markierungsphase weiß, kann es freigegeben werden.
	\item[grau:] Das Objekt ist erreichbar, allerdings wurden die Felder des Objekts noch nicht auf Referenzen zu weiteren Objekten überprüft.
	\item[schwarz:] Das Objekt ist erreichbar und alle Felder des Objekts wurden bereits überprüft.
\end{description}

Zu Beginn des Algorithmus sind alle existierenden Objekte weiß.
Der ursprüngliche Mark-Sweep-Algorithmus~\ref{algo:naive-mark} wird nun so modifiziert, dass Objekte bei ihrer Entdeckung grau und nach Abarbeitung ihrer Felder schwarz markiert werden.
Hierfür existiert eine atomare Prozedur \Method{setColor}, die die Markierung eines Objekts auf eine bestimmte Farbe \Var{WHITE}, \Var{GRAY} oder \Var{BLACK}\footnote{Die Farbinformation kann dadurch realisiert werden, dass jede Farbe mit einer eindeutigen Konstante identifiziert wird. Entsprechend ist darauf zu achten, dass dies mehr Speicherplatz zur Verwaltung der Markierungsinformationen benötigt.} setzt.
Auf diese Art bleiben nicht mehr erreichbare Objekte weiß und können anschließend als löschbar identifiziert werden.
Analog wird die \Method{sweep}-Prozedur in Algorithmus~\ref{algo:naive-sweep} so abgeändert, dass Objekte gelöscht werden, wenn sie weiß markiert sind.
Andernfalls wird ihre Markierung zurück auf weiß gesetzt.

\begin{algorithm}[h]
\begin{algorithmic}[1]
	\State \MethodHead{markStart}():
	\State \quad \Var{graySet} $\gets$ $\emptyset$				\Comment{Grau markierte Objekte}
	\State \quad \FOREACH \Var{obj} $\in \Roots$
	\State \quad \quad \IF \Method{isWhite}(\Var{obj})
	\State \quad \quad \quad \Method{setColor}(\Var{obj}, \Var{GRAY})	
	\State \quad \quad \quad \Method{add}(\Var{graySet}, \Var{obj})	
	\State \quad \quad \quad \Method{mark}()
	\Statex
	\State \MethodHead{mark}():
	\State \quad \WHILE \Var{graySet} $\neq \emptyset$
	\State \quad \quad \Var{obj} $\gets$ \Method{remove}(\Var{graySet})
	\State \quad \quad \Method{setColor}(\Var{obj}, \Var{BLACK})		\Comment{Objekt wird nun abgearbeitet}
	\State \quad \quad \FOREACH \Var{ref} $\in \Pointers(\Var{obj})$
	\State \quad \quad \quad \IF \Method{isWhite}(\Var{*ref})
	\State \quad \quad \quad \quad \Method{setColor}(\Var{*ref}, \Var{GRAY})	\Comment{Referenzierte Objekte grau markieren}
	\State \quad \quad \quad \quad \Method{add}(\Var{graySet}, \Var{*ref})
	\Statex
	\State \MethodHead{sweep}():
	\State \quad \Var{pos} $\gets$ \Method{nextObject}(\Var{HEAP\_START})
	\State \quad \WHILE \Var{pos} $\neq$ \Null
	\State \quad \quad \IF \Method{isWhite}(\Var{*pos})
	\State \quad \quad \quad \Method{free}(\Var{pos})
	\State \quad \quad \ELSE \Method{setColor}(\Var{*pos}, \Var{WHITE})
	\State \quad \quad \Var{pos} $\gets$ \Method{nextObject}(\Var{pos})
\end{algorithmic}
\caption[Markierung mit Drei-Farben-Abstraktion]{Markierung mit Drei-Farben-Abstraktion (vgl. \cite[S. 970]{dijkstra1978})}
\label{algo:tricolor}
\end{algorithm}

Mithilfe dieser \textbf{Drei-Farben-Abstraktion} (engl. \textit{tri-color abstraction}) können nun Referenzmanipulationen zugelassen werden, die parallel zur Markierungsphase der Garbage Collection stattfinden:
Wird in einem Objekt \Var{a} eine Referenz auf ein Objekt \Var{b} hinterlegt, so kann dies dazu führen, dass \Var{b} erreichbar wird.
Infolgedessen müssen auch alle von \Var{b} referenzierten Objekte als erreichbar identifiziert werden.
Somit ist es zielführend, \Var{b} beim Setzen der Referenz grau zu markieren und zu \Var{graySet} hinzuzufügen, sofern dies noch nicht der Fall ist oder die Felder von \Var{b} nicht bereits verfolgt wurden.
Dies kann etwa mit einer \textit{Schreibbarriere} (engl. \textit{write barrier}) realisiert werden, die genau dann zum Einsatz kommt, wenn während eines Kollektionszyklus eine Referenz in ein Feld eines Objektes geschrieben wird (siehe Algorithmus~\ref{algo:tricolor-barrier}).
Auf diese Art kann es jedoch vorkommen, dass Objekte unerreichbar werden, nachdem sie bereits grau oder schwarz markiert wurden.
Entsprechend verbleiben sie zunächst im Speicher und werden erst im nächsten Garbage-Collection-Zyklus entfernt.

\begin{algorithm}[h]
\begin{algorithmic}[1]
	\State \Atomic \MethodHead{writeRef}(\Var{obj}, \Var{i}, \Var{ref}):
	\State \quad \IF \Method{isWhite}(\Var{*ref})
	\State \quad \quad \Method{setColor}(\Var{*ref}, \Var{GRAY})
	\State \quad \quad \Method{add}(\Var{grayList}, \Var{*ref})
	\State \quad \Var{obj[i]} $\gets$ \Var{ref}
\end{algorithmic}
\caption[Schreibbarriere zur Manipulation von Referenzen]{Schreibbarriere zur Manipulation von Referenzen in Objekten. \Var{obj} bezeichnet das Objekt, in das die Referenz geschrieben wird, \Var{i} den Index des Feldes und \Var{ref} die zu schreibende Referenz.}
\label{algo:tricolor-barrier}
\end{algorithm}

An dieser Stelle gehen wir auf die Terminierung und Korrektheit des modifizierten Mark-Sweep-Algorithmus ein.

\begin{mybox}
\begin{satz}[Korrektheit der Drei-Farben-Abstraktion]
	\label{satz:tricolor-correctness}
	Der Mark-Sweep-Algorithmus mit Drei-Farben-Abstraktion terminiert und ist korrekt, sofern während der Arbeit des Kollektors keine neuen Objekte erzeugt werden.
\end{satz}
\end{mybox}

\begin{proof}[Beweis der Terminierung:]
	Zunächst ist festzuhalten, dass Objekte während der Markierungsphase ausschließlich \textit{dunkler} gefärbt werden:
	In Algorithmus~\ref{algo:tricolor} werden lediglich weiß markierte Objekte grau gefärbt (Zeile~5 und 14); eine Weißfärbung geschieht in dieser Phase grundsätzlich nicht.
	Analog zum ursprünglichen Algorithmus~\ref{algo:naive-mark} enthält \Var{graySet} nur grau markierte Objekte.
	Jedes Objekt befindet sich dadurch höchstens einmal in dieser Menge.
	Da \Var{graySet} nach jeder Iteration der \WHILE-Schleife um ein Objekt reduziert wird und keine neuen Objekte erzeugt werden, terminieren auch hier \WHILE- und \FOREACH-Schleife nach endlich vielen Schritten.
	Zuletzt terminiert auch die Bereinigungsphase (siehe Satz~\ref{satz:mark-sweep-correctness}).
	
	\textit{Zur Korrektheit:} Erneut ist zu zeigen, dass keine Objekte unmarkiert bleiben, die erreichbar sind.
	Dies wäre genau dann der Fall, wenn nach der Markierungsphase ein schwarz markiertes Objekt eine Referenz auf ein weiß markiertes Objekt besitzt.
	Daher ist die Gültigkeit folgender Schleifeninvariante für die \WHILE-Schleife zu zeigen:
	\begin{description}
		\item[(A)] Es existieren keine zwei Objekte \Var{a} und \Var{b} mit \Var{a} $\rightarrow$ \Var{b}, sodass \Var{a} schwarz markiert und \Var{b} weiß markiert ist.
	\end{description}
	Da zu Beginn alle Objekte weiß markiert sind und bis zum Eintritt in die \WHILE-Schleife Objekte nur grau markiert werden, gilt (A) trivialerweise, da keine schwarz markierten Objekte existieren.
	Gelte nun (A) zu Beginn einer Schleifeniteration und sei $\Var{obj} = \Var{a}$ dasjenige Objekt, das der Menge \Var{graySet} entnommen und schwarz gefärbt wird (Zeile~10f).
	Existiert nun ein weiteres Objekt \Var{b} mit \Var{a} $\rightarrow$ \Var{b}, so ist \Var{\&b} $\in \Pointers(\Var{a})$.
	Ist \Var{b} bereits grau oder schwarz markiert, so geschieht nichts.
	Andernfalls wird \Var{*(\&b)} $=$ \Var{b} grau markiert (Zeile~14).
	In beiden Fällen ist \Var{b} am Ende der Schleife nicht weiß markiert.
	Da \Var{a} das einzige Objekt ist, das in dieser Iteration schwarz gefärbt wird, ist (A) nach der Iteration weiterhin gültig.
	Da die Schleife terminiert, die Invariante aufrecht erhalten bleibt und in der Bereinigungsphase nur weiß markierte Objekte freigegeben werden, ist der Algorithmus korrekt.
\end{proof}

Die Atomizität der Schreibbarriere \Method{writeRef} in Algorithmus~\ref{algo:tricolor-barrier} ist in der Tat notwendig, um die Korrektheit zu gewährleisten.
Käme es unmittelbar nach der Markierung des Referenzziels, aber vor dem eigentlichen Setzen der Referenz in Zeile 5, zu einen kompletten Garbage-Collection-Zyklus, so würde die Markierung durch den Kollektor wieder aufgehoben.
Damit bestünde im Anschluss die Möglichkeit, dass die Invariante (A) verletzt wird.
Allerdings kann die Invariante (A) abgeschwächt und damit eine \textit{feingranularere} Lösung ermöglicht werden.
Für Details hierzu sei auf die Publikation von \textsc{Dijkstra} et al. verwiesen \cite[S. 972ff]{dijkstra1978}.
\textsc{Pirinen} liefert darüber hinaus einen Überblick über Vor- und Nachteile verschiedener Lese- und Schreibbarrieren im Kontext der Drei-Farben-Abstraktion \cite{pirinen}.


% % % % % % % % % % % % % % % % % % % % % % % % % % % % % 
 % % % % % % % % % % % % % % % % % % % % % % % % % % % % % 
% % % % % % % % % % % % % % % % % % % % % % % % % % % % % 


\section{Verzögerte Bereinigung}
\label{sec:lazy-sweep}
Die im vorigen Abschnitt vorgestellte Drei-Farben-Markierung dient vor allem dazu, die durch die Markierungsphase entstehenden Verzögerungen im Programmablauf zu reduzieren.
In diesem Abschnitt wird eine Reduktion der Kosten der Sweep-Phase angestrebt.
Meist können Mutator und Kollektor während der Bereinigung gleichzeitig tätig sein:
Nicht mehr erreichbare Objekte sind ohne Nebenwirkungen zerstörbar und Markierungsinformationen werden so hinterlegt, dass sie für den Mutator grundsätzlich nicht zugänglich sind.
\textit{Race conditions} sind dadurch prinzipiell ausgeschlossen.
Ferner ergibt sich hierdurch die Möglichkeit, die Bereinigungsphase nicht unmittelbar nach der Markierungsphase ausführen zu müssen.
Für Systeme ohne hardware- oder softwareseitige Unterstützung für Multithreading, in denen die Aufgaben des Mutators und Kollektors nicht gleichzeitig bearbeitet werden können, bietet sich hingegen die \textbf{verzögerte Bereinigung} (engl. \textit{lazy sweeping}) nach \textsc{Hughes} \cite{hughes} an.

\begin{figure}[h]
	\centering
	\includestandalone{img/tikz/ch2-heap-region}
	\caption[Veranschaulichung eines in Blöcken aufgeteilten Heaps]{Veranschaulichung eines in Blöcken aufgeteilten Heaps. Jeder Block kann Objekten einer festgelegten Größenordnung zugewiesen werden. Für eine Größenordnung können allerdings mehrere Blöcke verwendet werden.}
	\label{fig:heap-region}
\end{figure}

Bei der verzögerten Bereinigung ist der Heapspeicher in \textit{Blöcken} eingeteilt.
Jeder Block enthält Objekte einer festgelegten Größenordnung; zu jeder Größenordnung können mehrere Blöcke existieren (Abbildung~\ref{fig:heap-region}).
Die Idee ist nun, das Sweeping durch den Allokator ausführen zu lassen, wenn Speicher angefordert wird.
Dabei werden die zuvor markierten Objekte freigegeben, allerdings nur in denjenigen Blöcken, die der angeforderten Speichermenge zugeordnet sind.
Die Bereinigung beschränkt sich dadurch auf einen Bruchteil des Heaps.

\begin{algorithm}[h]
\begin{algorithmic}[1]
	\State \MethodHead{new}(\Var{size}):
	\State \quad \Var{adr} $\gets$ \Method{allocate}(\Var{size})
	\State \quad \IF \Var{adr} $=$ \Null
	\State \quad \quad \Method{lazySweep}(\Var{size})	\Comment{Starte Lazy-Sweep-Zyklus}
	\State \quad \quad \Var{adr} $\gets$ \Method{allocate}(\Var{size})	\Comment{Zweiter Versuch}
	\State \quad \quad \IF \Var{adr} $=$ \Null
	\State \quad \quad \quad \Method{collect}()		\Comment{Nach weiteren unerreichbaren Objekten suchen}
	\State \quad \quad \quad \Method{lazySweep}(\Var{size})	\Comment{Zweiter Lazy-Sweep-Zyklus}
	\State \quad \quad \quad \Var{adr} $\gets$ \Method{allocate}(\Var{size})	\Comment{Dritter Versuch}
	\State \quad \quad \quad \IF \Var{adr} $=$ \Null
	\State \quad \quad \quad \quad \Method{error}(\Var{"Nicht genügend Speicher"})
	\State \quad \Return \Var{adr}
	\Statex
	\State \MethodHead{collect()}:
	\State \quad \Method{markStart()}		\Comment{Siehe Algorithmus~\ref{algo:naive-sweep} bzw. \ref{algo:tricolor}}
	\State \quad \FOREACH \Var{blk} $\in \Blocks$
	\State \quad \quad \IF \NOT \Method{isMarked}(\Var{blk})
	\State \quad \quad \quad \Method{free}(\Var{blk})	\Comment{Gib komplett verwaisten Block sofort frei}
	\State \quad \quad \ELSE \Method{add}(\Var{toSweep}, \Var{blk})
	\Statex
	\State \MethodHead{lazySweep}(\Var{size})
	\State \quad \Do
	\State \quad \quad \Var{blk} $\gets$ \Method{remove}(\Var{toSweep}, \Var{size})
	\State \quad \quad \IF \Method{sweep}(\Method{start}(\Var{blk}), \Method{end}(\Var{blk}))
	\State \quad \quad \quad \Return	\Comment{Beende, sobald Speicher gewonnen wurde}
	\State \quad \WHILE \Var{blk} $\neq$ \Var{null}
	\State \quad \Method{createBlock}(\Var{size})	\Comment{Initialisiere neuen Block, wenn nichts freigegeben}
\end{algorithmic}
\caption[Verzögertes Bereinigen des Heaps]{Verzögertes Bereinigen des Heaps (vgl. \cite[S. 25]{handbook}).}
\label{algo:lazy-sweep}
\end{algorithm}

Dieser Ansatz geht mit einigen marginalen Änderungen an den bisher betrachteten Algorithmen einher.
Zum einen wird vorausgesetzt, dass beim Markieren eines Objekts auch der entsprechende Block markiert wird, in dem sich das Objekt befindet.
Dies kann recht einfach realisiert werden, indem bei der Initialisierung eines Blocks durch den Allokator zusätzlich Platz für einen Header reserviert wird, welcher Metadaten wie die Größenordnung der enthaltenen Objekte aufnimmt.
Weiter wird verlangt, dass der Allokator zusätzlich Informationen über Anzahl, Lage und Größenordnung der existierenden Blöcke besitzt.
Zuletzt wird die Prozedur \Method{sweep} aus den Algorithmen~\ref{algo:naive-sweep} und \ref{algo:tricolor} dahingehend abgewandelt, dass sie nur einen eingeschränkten Teil des Heaps traversiert -- etwa, indem zwei Parameter für Start- und Endadresse übergeben werden.
Zudem soll \Method{sweep} den Wahrheitswert \Var{true} zurückgeben, wenn zumindest ein Objekt entfernt wurde, und andernfalls \Var{false}.

Nach Abschluss der Markierungsphase kann anhand der zusätzlichen Markierung der Blöcke festgestellt werden, welche nur verwaiste Objekte enthalten.
In diesem Fall kann der Allokator einen Teilbereich des Heaps en bloc freigeben (Zeile~17 von Algorithmus~\ref{algo:lazy-sweep}).
Hingegen werden Blöcke, die mindestens ein erreichbares Objekt enthalten und somit markiert sind, nach Abschluss der Markierungsphase einer Menge \Var{toSweep} zugefügt (Zeile~18).
In dieser wird Buch geführt, welche Blöcke differenzierter bereinigt werden müssen, da sie sowohl erreichbare als auch unerreichbare Objekte enthalten können.

Im Gegensatz zu den bisher betrachteten Mark-Sweep-Algorithmen beginnt die Bereinigungsphase nicht unmittelbar nach Abschluss der Markierung, sondern erst bei Anforderung von Speicher mittels \Method{new}.
Falls dabei kein freier Speicher geeigneter Größe gefunden werden kann, weil etwa die entsprechenden Blöcke ausgeschöpft sind, beginnt ein \textit{Lazy-Sweep-Zyklus} (Zeile~20 bis~24).
Dabei werden nach und nach Blöcke aus \Var{toSweep} entfernt und bereinigt, die Objekte der angeforderten Speichergröße verwalten.
Sobald ein Block um ein Objekt reduziert werden konnte, wird der Zyklus beendet und der Allokator kann den gewonnenen Speicher neu zuordnen (Zeile~22f).
Falls aus keinem Block ein Objekt freigegeben werden konnte, bedeutet dies, dass alle Blöcke der entsprechenden Objektgröße mit erreichbaren Objekten gefüllt sind.
In diesem Fall muss der Allokator einen neuen Block initialisieren, der neue Objekte aufnehmen kann (Zeile~25).

Was geschieht nun, wenn nicht genügend freier Speicher zu Verfügung steht, um einen weiteren Block zu erzeugen?
In diesem Fall schlägt auch der zweite Al"-lo"-ka"-tions"-versuch fehl (Zeile~5f).
Der letzte Versuch besteht nun darin, die Markierungen der Objekte zu aktualisieren, indem eine Markierungsphase gestartet wird (Zeile~7).
Da die letzte Markierungsphase nicht unmittelbar vor der Speicheranforderung, sondern möglicherweise wesentlich früher stattgefunden hat, könnten in der Zwischenzeit weitere Objekte verwaist sein.
Diese werden nachfolgend in einem zweiten Lazy-Sweep-Zyklus freigegeben, sodass zuletzt Platz innerhalb eines Blocks bzw. für einen neuen Block geschaffen werden kann (Zeile~8).

Während die Einschränkung der Bereinigung auf einzelne Blöcke zwar einen Performancevorteil mit sich bringt, besitzt diese Variante des Mark-Sweep-Algorithmus auch einige Nachteile:
Eine ungünstige Kombination von Speicheranforderungen und Sweep-Zyklen könnte dazu führen, dass viele Blöcke einer Größenordnung angelegt werden, die jeweils nur wenige Objekte enthalten und größtenteils ungenutzt sind.
Wenn jeder Block allerdings mindestens ein erreichbares Objekt enthält, kann er nicht in Gänze freigegeben werden.
Eine hohe Zahl dieser ineffizient genutzter Blöcke führt dazu, dass möglicherweise nicht mehr genügend freier Speicher zu Verfügung steht, um Blöcke anderer Größenordnung zu initialisieren, obwohl ausreichend Speicher vorhanden wäre, der nicht durch Objekte belegt wird.
Eine Lösung dieses Problems wäre eine zusätzliche Konsolidierungsphase, bei der Blöcke gleicher Größenordnung zusammengelegt werden.
Dies bedeutet jedoch zusätzlicher Aufwand für die Verschiebung von Objekten (siehe Kapitel~\ref{cha:compacting}).
Der zweite Nachteil ist, dass nicht mehr erreichbare Objekte eventuell erst sehr spät freigegeben werden, wenn primär Objekte anderer Größenordnungen angefordert werden.
In so einem Fall können auch Blöcke, die keine erreichbare Objekte mehr enthalten, über lange Zeit im Speicher verweilen, sodass diese für einen gewissen Zeitraum ein Speicherleck darstellen.
Zuletzt kann auch die Aufteilung des Heaps nach Objektgrößen nicht zielführend sein:
Je nach Anwendungsfall und verwendeter Programmiersprache besteht die Möglichkeit, dass zwischen den meisten Objekten keine signifikanten Größenunterschiede bestehen und die Blöcke einer bestimmten Größenordnung besonders stark beansprucht werden.
Der Vorteil, nur einen beschränktes Gebiet des Heaps zu bereinigen, würde damit deutlich geschmälert werden.


% % % % % % % % % % % % % % % % % % % % % % % % % % % % % 
 % % % % % % % % % % % % % % % % % % % % % % % % % % % % % 
% % % % % % % % % % % % % % % % % % % % % % % % % % % % %


\section{Weitere Varianten und Komplexität}
\label{sec:mark-sweep-variations}
Der in diesem Kapitel betrachtete Mark-Sweep-Algorithmus gilt als der erste Algorithmus zur automatischen Speicherverwaltung, weswegen zahlreiche Varianten existieren, die für bestimmte Anwendungen und Systeme optimiert sind.
Allen gemein ist das Auffinden erreichbarer Objekte durch Verfolgung von Referenzen in der Markierungsphase.
Obwohl die erreichbaren Objekte weniger Speicher belegen als während der Bereinigungsphase traversiert werden muss, ist die Markierungsphase in der Praxis die aufwendigere.
Häufig befinden sich Objekte, die aufeinander verweisen, nicht in unmittelbarer Nähe zueinander.
Das Auffinden erreichbarer Objekte verursacht daher schwer vorhersehbare Speicherzugriffe.
Caching- und Prefetch-Mechanismen, die durch vorweggenommenes Bereitstellen von Speicherinhalten Zugriffe beschleunigen, profitieren hingegen von \textit{zeitlicher} und \textit{räumlicher Lokalität} von Daten, die in einer Beziehung zueinander stehen.
In der Markierungsphase können sie daher Speicherzugriffe nicht so effektiv beschleunigen wie in der Bereinigungsphase \cite[S. 21f]{handbook}.
Um diesen Makel zu beheben, kann die Zugriffsreihenfolge auf Objekte variiert werden.
Anstatt etwa im naiven Algorithmus~\ref{algo:naive-mark} die Abarbeitung der \Var{toDo}-Menge zu beginnen, sobald das erste Basisobjekt erfasst wurde, können stattdessen auch zunächst alle Basisobjekte zu \Var{toDo} hinzugefügt und die Prozedur \Method{mark} im Anschluss aufgerufen werden.
Je nachdem, wie \Var{toDo} in der Praxis realisiert wird -- zum Beispiel in Form eines Stacks -- kann die Traversierung der Objekte -- und damit die Performanz des Kollektors in der Markierungsphase -- signifikant beeinflusst werden (vgl. \cite[S. 19]{handbook}).

Statt Markierungsinformationen im Header zu speichern, können diese auch in Bitmaps oder tabellenähnlichen Strukturen außerhalb eines Objekts verwaltet werden.
Dies vermeidet schreibende Zugriffe auf Objekte während der Markierungsphase, sodass Objekte ohne Referenzfelder übersprungen werden können und die Garbage Collection das Caching weniger stark beeinträchtigt.
Im Falle eines in Blöcken eingeteilten Heaps bietet es sich beispielsweise an, pro Block eine Bitmap anzulegen.
Somit kann während der Bereinigung schnell erkannt werden, ob größere Teile eines Blocks freigegeben werden können, ohne ihn komplett traversieren zu müssen.

Dieses Kapitel schließt mit einer Betrachtung der Komplexität der vorgestellten Mark-Sweep-Varianten ab.
In seiner einfachsten Form erweist sich der Mark-Sweep-Algorithmus als vergleichsweise platzsparend:
Werden Markierungsinformationen unmittelbar in den Objekten gespeichert, wird pro Objekt lediglich ein einzelnes Bit oder Byte beansprucht.
Zwischen verschiedenen Mark-Sweep-Zyklen müssen ferner keinerlei zusätzliche Informationen bereitgehalten werden.
Während eines yklus muss jedoch die Menge \Var{toDo} der noch zu untersuchenden Objekte verwaltet werden.
Da bereits festgestellt wurde, dass keine Objekte mehrfach zu \Var{toDo} hinzugefügt werden, ist ihre Mächtigkeit durch die Anzahl $|\Reach|$ der erreichbare Objekte beschränkt.
Davon ausgehend, dass in \Var{toDo} ausschließlich Referenzen konstanter Größe gespeichert werden müssen, ergibt sich insgesamt ein Platzbedarf von $\oh(|\Reach|)$.

Während der Speicherbedarf der Markierungsphase also linear mit der Anzahl der erreichbaren Objekte wächst, spielt für die Laufzeit auch die Anzahl der Referenzen eine Rolle.
Für jedes erreichbare Objekt müssen alle enthaltenen Referenzen verfolgt werden, um auszuschließen, dass ein erreichbares Objekt nicht unmarkiert bleibt.
Die Komplexität dieser Operation ist analog zur vollständigen Traversierung eines beliebigen Graphen $(V,E)$, die durch $\oh(|V|+|E|)$ gegeben ist \cite[Kap. 22]{cormen-leiserson}.

\begin{mybox}
\begin{satz}[Speicherbedarf und Komplexität des Mark-Sweep-Algorithmus]
	\label{satz:mark-sweep-complexity}
	Für den naiven Mark-Sweep-Algorithmus gelten folgende Eigenschaften:
	\begin{enumerate}[(1)]
		\item Der Speicherbedarf des gesamten Algorithmus ist $\oh(|\Reach|)$.
		\item Die Laufzeit der Markierungsphase ist $\oh\enb{|\Reach|+\sum\limits_{\Var{a} \in \Reach}^{} \abs{\Pointers(\Var{a})}}$.
		\item Die Laufzeit der Bereinigungsphase ist $\oh(|\HeapSize|)$, wobei $|\HeapSize|$ die Größe des Heaps bezeichnet.
	\end{enumerate}
\end{satz}
\end{mybox}

Für die Drei-Farben-Abstraktion ergeben sich identische asymptotische Laufzeiten; in der Praxis ermöglicht die Nebenläufigkeit von Kollektor und Mutator jedoch geringere Verzögerungen im Programmablauf.
Die Kosten der Bereinigungsphase des Lazy Sweeping nach \textsc{Hughes} sind nur schwer abschätzbar, da sie unter anderem von der Strukturierung des Heaps in Blöcken abhängen, welche wiederum je nach Anwendungsfall einen starken Einfluss auf die Ausführungshäufigkeit der Garbage Collection haben kann.
Grundsätzlich ist es für Mark-Sweep-Ansätze empfehlenswert, einen größeren Puffer für die Arbeit des Kollektors zu reservieren, um eine hohe Zahl an Speicheranforderungen seitens des Allokators bewältigen zu können.
Andernfalls besteht die Gefahr, dass die Garbage Collection zu häufig ausgelöst wird und den Mutator nach und nach verdrängt (vgl. \cite[S. 70]{jones-lins}).