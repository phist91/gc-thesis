%!TEX root = ../../thesis.tex
\chapter{Mark and Sweep}
\label{cha:mark-sweep}
Wir beginnen mit einer Vorstellung des ersten Garbage-Collection-Algorithmus, der auf John \textsc{McCarthy} zurückgeht \cite[191--193]{mccarthy1960}.
Im Rahmen eines im Jahr 1960 veröffentlichten Artikels über die Berechnung rekursiver Funktionen auf dem \textit{IBM 704} mithilfe des \textit{LISP Programming Systems} erläutert McCarthy die Speicherung von Daten in einer Listenstruktur.
Diese besteht aus Paaren, deren erster Eintrag \texttt{car} die zu speichernde Information enthält, während im zweiten Eintrag \texttt{cdr} die Registeradresse des nachfolgenden Paares zu finden ist.
Register, die aktuell nicht zur Speicherung von Daten genutzt werden, befinden sich in einer \textit{free storage list}.
Bei der Anforderung von Speicher für ein zu speicherndes Datum werden Register aus dieser Liste entfernt.
Durch die Manipulation der Registeradressen können Paare verwaisen, was zu Speicherlecks führt.
Zur Auflösung dieser Problematik bietet LISP als erste Programmiersprache ihrer Zeit eine automatische Speicherverwaltung, die von McCarthy wie folgt grob umschrieben wird:
Im Falle von Speicherknappheit wird -- ausgehend von einer Menge von Basisregistern -- ermittelt, welche Register über eine Folge von \texttt{cdr}-Einträgen erreichbar sind.
Nicht erreichbare Register enthalten überschreibbare Inhalte, sodass diese zurück in die \textit{free storage list} eingefügt werden können und wieder als freie Speicherplätze zu Verfügung stehen.
Diese zweischrittige Vorgehensweise -- das Erkennen nicht mehr benötigter Speicherbereiche und die anschließende Freigabe eben jener -- bildet die Grundlage des \textit{Mark-and-Sweep}-Algorithmus.

\section{Naives Mark and Sweep}
\label{sec:naive-mark-sweep}