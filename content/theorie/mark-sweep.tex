%!TEX root = ../../thesis.tex
\chapter{Mark and Sweep}
\label{cha:mark-sweep}
Wir beginnen mit einer Vorstellung des ersten Garbage-Collection-Algorithmus, der auf John \textsc{McCarthy} zurückgeht \cite[191--193]{mccarthy1960}.
Im Rahmen eines im Jahr 1960 veröffentlichten Artikels über die Berechnung rekursiver Funktionen auf dem \textit{IBM 704} mithilfe des \textit{LISP Programming Systems} erläutert McCarthy die Speicherung von Daten in einer Listenstruktur.
Diese besteht aus Paaren, deren erster Eintrag \texttt{car} die zu speichernde Information enthält, während im zweiten Eintrag \texttt{cdr} die Registeradresse des nachfolgenden Paares zu finden ist.
Register, die aktuell nicht zur Speicherung von Daten genutzt werden, befinden sich in einer \textit{free storage list}.
Bei der Anforderung von Speicher für ein zu speicherndes Datum werden Register aus dieser Liste entfernt.
Durch die Manipulation der Registeradressen können Paare verwaisen, was zu Speicherlecks führt.
Zur Auflösung dieser Problematik bietet LISP als erste Programmiersprache ihrer Zeit eine automatische Speicherverwaltung, die von McCarthy wie folgt grob umschrieben wird:
Im Falle von Speicherknappheit wird -- ausgehend von einer Menge von Basisregistern -- ermittelt, welche Register über eine Folge von \texttt{cdr}-Einträgen erreichbar sind.
Nicht erreichbare Register enthalten überschreibbare Inhalte, sodass diese zurück in die \textit{free storage list} eingefügt werden können und wieder als freie Speicherplätze zu Verfügung stehen.
Diese zweischrittige Vorgehensweise -- das Erkennen nicht mehr benötigter Speicherbereiche und die anschließende Freigabe eben jener -- bildet die Grundlage des \textit{Mark-and-Sweep}-Algorithmus.

\section{Naives Mark and Sweep}
\label{sec:naive-mark-sweep}
Der naive Mark-and-Sweep-Algorithmus arbeitet in zwei Schritten:
Zunächst wird bestimmt, welche Objekte im Speicher unerreichbar sind, weil sie von keinem anderen erreichbaren Objekt referenziert werden.
Diese Objekte können gefahrlos freigegeben werden, da auf ihre Informationen nicht mehr zugegriffen werden kann.
Der zweite Schritt besteht aus einer Traversierung des gesamten Heaps.
Dabei werden alle existierenden Objekte besucht und diejenigen freigegeben, die im ersten Schritt als unerreichbar identifiziert werden konnten.
Die entsprechenden Speicherbereiche stehen anschließend wieder für neue Objekte zu Verfügung.

\begin{algorithm}
\begin{algorithmic}[1]
	\State \MethodHead{collectGarbage}():
	\State \quad \Method{markStart}()	\Comment{Starte Markierung}
	\State \quad \Method{sweepHeap}()	\Comment{gesamten Heap traversieren}
	\State
	\State \MethodHead{markStart}():
	\State \quad \Var{toDo} $\gets$ $\emptyset$				\Comment{Noch abzuarbeitende Objekte}
	\State \quad \FOREACH \Var{obj} $\in \Roots$		\Comment{Beginne mit Basisobjekten}
	\State \quad \quad \IF (isNotMarked(\Var{obj}))
	\State \quad \quad \quad \Method{setMarked}(\Var{obj})	\Comment{Objekt als erreichbar markieren}
	\State \quad \quad \quad \Method{add}(\Var{toDo}, \Var{obj})	\Comment{Objekt zu ToDo-Liste hinzufügen}
	\State \quad \quad \quad \Method{mark}()			\Comment{Abarbeitung starten}
	\State
	\State \MethodHead{mark}():
	\State \quad \WHILE \Var{toDo} $\neq \emptyset$
	\State \quad \quad obj $\gets$ \Method{remove}(\Var{toDo})			\Comment{Hole nächstes abzuarbeitendes Objekt}
	\State \quad \quad \FOREACH \Var{adr} $\in \Pointers(\Var{obj})$	\Comment{Hole alle referenzierten Objekte\dots}
	\State \quad \quad \quad \IF (\Var{adr} $\neq$ \Null \&\& \Method{isNotMarked}(\Var{*adr}))	\Comment{\dots die unmarkiert sind}
	\State \quad \quad \quad \quad \Method{setMarked}(\Var{*adr})	\Comment{Markiere diese und füge zu ToDo-Liste hinzu}
	\State \quad \quad \quad \quad \Method{add}(toDo, \Var{*adr})
\end{algorithmic}
\caption[Naives Mark and Sweep]{Naives Mark and Sweep \cite[Kap. 2.2]{jones-lins}}
\label{algo:naive-mark-sweep}
\end{algorithm}

Die Markierungsphase (engl. \textit{mark}) funktioniert wie folgt:
Zunächst wird mittels der Methode \Method{markStart} eine Menge \Var{toDo} erzeugt, die diejenigen Objekte enthält, die bereits als erreichbar erkannt wurden, aber selbst noch nicht verarbeitet wurden (Zeile 6 in Algorithmus~\ref{algo:naive-mark-sweep}).
In diese werden alle bislang unmarkierten Basisobjekte der Menge \Roots eingefügt und markiert, da sie in jedem Fall erreichbar sind (Zeile 7 bis 10).
Ist ein Basisobjekt bereits markiert worden, so wurde es schon entdeckt -- etwa, weil es durch ein zuvor abgearbeitetes Objekt referenziert wird.
Daraus folgt, dass es ebenfalls bereits abgearbeitet wurde oder sich noch in der Menge \Var{toDo} befindet.
In beiden Fällen muss es folglich nicht erneut zu \Var{toDo} hinzugefügt werden.

Bereits nach dem Hinzufügen des ersten Basisobjekts wird die Methode \Method{mark} aufgerufen, welche die \Var{toDo}-Menge abarbeitet.
Für jedes Objekt in \Var{toDo} werden diejenigen Felder betrachtet, die eine Referenz auf ein Objekt enthalten (Zeile 15 und 16).
Wenn dieses Objekt noch nicht markiert wurde, wird es in diesem Augenblick zum ersten Mal entdeckt.
Da es somit erreichbar ist, kann es markiert und zu \Var{toDo} hinzugefügt werden, um zu einem späteren Zeitpunkt abgearbeitet zu werden (Zeile 17 und 18).
Verweist die Referenz hingegen auf ein Objekt, das bereits markiert wurde, wurde dieses schon zuvor entdeckt.
Auch hier ist ein erneutes Hinzufügen zu \Var{toDo} überflüssig.
Sobald \Var{toDo} leer ist, erfolgt die Rückkehr zur Methode \Method{markStart}, sodass ggfs. das nächste Basisobjekt abgearbeitet wird.

\todo[inline]{Kleines Beispiel hier, großes in den Anhang?}

Es ist wesentlich, dass Objekte bereits markiert werden, wenn sie der Menge \Var{toDo} hinzugefügt werden, und nicht etwa, nachdem sie abgearbeitet wurden (Zeile 9 und 10 bzw. 18 und 19).
Andernfalls besteht bei zyklischen Referenzen die Gefahr einer Endlosschleife, da unmarkierte Objekte mehrfach hinzugefügt würden.
Präziser können wir festhalten, dass \Var{toDo} zu jedem Zeitpunkt ausschließlich bereits markierte Objekte enthält.
Da keine Objekte hinzugefügt werden, die bereits markiert wurden (Zeile 8 und 16), wird kein Objekt doppelt verarbeitet.
Da zudem mit jeder Iteration der \WHILE-Schleife mindestens ein Objekt aus \Var{toDo} entfernt wird (Zeile 15), die Anzahl aller Objekte endlich ist und wir voraussetzen, dass während der Ausführung des Garbage Collectors keine neuen Objekte entstehen, wird sowohl die \WHILE-Schleife, als auch die \FOREACH-Schleife nach endlich vielen Schritten terminieren.

Anstatt die Abarbeitung der \Var{toDo}-Menge zu beginnen, sobald das erste Basisobjekt erfasst wurde, können statt dessen auch zunächst alle Basisobjekte zu \Var{toDo} hinzugefügt und die Methode \Method{mark} im Anschluss aufgerufen werden.
Je nachdem, wie \Var{toDo} in der Praxis realisiert wird -- zum Beispiel in Form eines Stacks -- kann damit die Traversierung der Objekte beeinflusst werden.
Dies kann einen erheblichen Einfluss auf die Performanz der Markierungsphase haben, wenn Caching-Effekte eine Rolle spielen. \todo{evtl. später genauer drauf eingehen}

