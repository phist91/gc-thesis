%!TEX root = ../../thesis.tex
\chapter{Referenzzählung}
\label{cha:reference-counting}
Der zweite Garbage-Collection-Algorithmus, der in dieser Arbeit vorgestellt wird, stammt von George Collins und aus dem selben Jahr wie der Mark-Sweep-Algorithmus.
Collins betrachtet wie McCarthy das \textit{LISP Programming System} auf IBM-Großrechnern und das Problem, wann Listenelemente, die prinzipiell in mehreren Listen vorkommen können, wieder freigegeben werden dürfen.
McCarthys Ansatz bezeichnet er jedoch als \enquote{elegant but inefficient} \cite[S. 655]{collins1960}, da er sowohl zeitaufwändig sei als auch den Speicherplatz für Nutzdaten einschränke.
Stattdessen schlägt Collins vor, zusätzlich zu einem Datum die Anzahl der Referenzen auf dieses zu speichern.
Anhand dieses Zählers, der bei der Manipulation von Referenzen aktualisiert wird, kann unmittelbar festgestellt werden, ob ein Datum verwaist ist und der entsprechende Speicherplatz freigegeben werden kann \cite[S. 656f]{collins1960}.
Diesen Ansatz -- die \textit{Referenzzählung} -- werden wir in diesem Kapitel ausführlich betrachten.