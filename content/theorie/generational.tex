%!TEX root = ../../thesis.tex
\chapter{Hybride und generationelle Ansätze}
\label{cha:generational}

In den vorigen Kapiteln wurden hauptsächlich Algorithmen betrachtet, deren Anwendung stets den gesamten Heap betrifft.
An zwei Stellen war jedoch zu erkennen, dass es zielführend sein kann, auf diesen Grundsatz zu verzichten:
Die verzögerte Bereinigung nach \textsc{Hughes} (Abschnitt~\ref{sec:lazy-sweep}) teilt den Heap nach Objekten gleicher Größe ein, um das Sweeping nur bei Bedarf auf einen Bruchteil des Heaps ausführen zu müssen.
Die Einführung von Handles in Abschnitt~\ref{sec:handle} legt die Speicherung eben jener in einem separaten Bereich nahe.
Da die Handles im Gegensatz zu den eigentlichen Objekten nicht verschoben werden, muss dieser Bereich mit einem anderen Garbage-Collection-Ansatz verwaltet werden.

Die Partitionierung des Heaps in Bereiche ermöglicht eine feingranularere Auswahl an Garbage-Collection-Strategien.
Die verschiedenen Bereiche können unterschiedlich häufig, durch passend abgestimmte Algorithmen und/oder eine Kombination verschiedener Ansätze bereinigt werden.
Im Folgenden werden wir daher zunächst auf diverse Kriterien zur Heappartitionierung eingehen, bevor anschließend eine Auswahl dieser hybriden Algorithmen betrachtet wird.


% % % % % % % % % % % % % % % % % % % % % % % % % % % % % % % % % % 
 % % % % % % % % % % % % % % % % % % % % % % % % % % % % % % % % % % 
% % % % % % % % % % % % % % % % % % % % % % % % % % % % % % % % % % 


\section{Algorithmus von Lieberman und Hewitt}
\label{sec:lieberman}
Eines der ersten Verfahren, der die Lebensdauer der Objekte als Kriterium für einen Kollektionszyklus nutzt, stammt von \textsc{Lieberman} und \textsc{Hewitt} \cite{lieberman1983}.
Der Heap wird dazu ebenfalls in Blöcke eingeteilt, die durch zwei Zahlen charakterisiert werden:
Die \textit{Generation} eines Blocks ist ein Indikator für die Lebensdauer der enthaltenen Objekte und wird regelmäßig inkrementiert, sodass zu jeder Generation genau ein Block gehört.
Blöcke, die jünger sind, besitzen somit eine höhere Generationenzahl und neue Objekte werden immer in der zuletzt erzeugten Generation gespeichert.
Die \textit{Version} eines Blocks gibt an, wie oft er durch die Garbage Collection bereinigt wurde.
Die Idee ist nun (analog zu Abschnitt~\ref{sec:lazy-sweep}), die Garbage Collection generationenweise arbeiten zu lassen und die Laufzeit des Algorithmus für jüngere Generationen möglichst kurz zu halten.
Dies geht zurück auf eine Hypothese von \textsc{Deutsch} und \textsc{Bobrow}, wonach ein Großteil der Objekte entweder frühzeitig verwaist oder langfristig erreichbar bleibt (vgl. \cite[S. 523]{deutsch1976}).
Um dies zu erreichen, wird die Konvention eingeführt, dass nur direkte Referenzen innerhalb einer Generation oder von Objekten einer jüngeren Generation auf solche einer älteren erlaubt sind.
Eine Referenz $\Var{a} \rightarrow \Var{b}$ eines älteren Objekts \Var{a} auf ein jüngeres Objekt \Var{b} wird indirekt über einen Handle realisiert, welcher in einem separaten Bereich der Generation von \Var{b} gespeichert wird und -- neben einer internen Referenz auf \Var{b} -- zusätzlich eine \textit{Rückreferenz} \Var{origin} auf das entsprechende Feld von \Var{a} besitzt (siehe Abbildung~\ref{fig:lieberman-generations})\footnote{Wir gehen im Folgenden davon aus, dass jeder Handle eine einzelne Referenz verwaltet. Natürlich können auch alle Referenzen auf \Var{b} über einen einzigen Handle realisiert werden; dieser muss entsprechend Rückreferenzen auf alle Felder speichern, die \Var{b} referenzieren. Der Algorithmus ist entsprechend anzupassen.}.

\begin{figure}[h]
	\centering
	\includestandalone[scale=1]{img/tikz/ch5-lieberman1}
	\caption[Beispielhafte Objektkonstellation zwischen drei Generationen]{Beispielhafte Objektkonstellation zwischen drei Generationen. Der Zähler unten rechts gibt die akuelle Version einer Generation an. Neben Referenzen von jüngeren auf ältere Objekten existieren auch interne Referenzen zwischen gleich alten Objekten (\textcolor{ctcolormain}{blau}) und indirekte Referenzen von älteren auf jüngere Objekte über Handles (\textcolor{ctcoloraccessory}{purpur}).}
	\label{fig:lieberman-generations}
\end{figure}

Der Kollektor geht bei der Bereinigung der Generation $k$  (\Var{source}) in mehreren Schritten vor (siehe Algorithmus~\ref{algo:lieberman}):
Als erstes wird ein Block für eine neue Version \Var{target} der betrachteten Generation angelegt (Zeile~3).
Nun werden die erreichbaren Objekte in \Var{source} nach \Var{target} kopiert, vergleichbar mit dem Halbraumverfahren aus Abschnitt~\ref{sec:copying}.
Zunächst werden dazu alle Objekte kopiert, die über Handles erreichbar sind (Prozedur \Method{updateHandles}).
Dabei wird für jeden auffindbaren Handle in \Var{source} das entsprechende Ziel nach \Var{target} kopiert, ein neuer Handle in \Var{target} angelegt und das Feld des älteren Objekts angepasst, das eine Referenz auf den ursprünglichen Handle enthält (Zeile~12 bis 14).
Weiter werden alle direkten Referenzen auf Objekte der $k$-ten Generation durch die Prozedur \Method{updateFields} aktualisiert.
Aufgrund der oben eingeführten Konvention genügt es, alle Blöcke mit höherer Generationenzahl zu untersuchen (Zeile~16).
Auch hier werden dann Referenzen auf Objekte aus der $k$-ten Generation nach \Var{target} kopiert und die korrespondierenden Felder angepasst.

\begin{algorithm}[h!]
\begin{algorithmic}[1]
	\State \MethodHead{collectGarbage}(\Var{k}):
	\State \quad $\Var{source} \gets \textsc{Gen}_k$
	\State \quad $\Var{target} \gets \Method{newVersion}(\textsc{Gen}_k)$		\Comment{Neue Version der Generation initialisieren}
	\State \quad $\textsc{Gen}_k \gets \Var{target}$
	\State \quad $\Var{pos} \gets \Var{target}$
	\State \quad \Method{updateHandles}(\Var{source},\Var{target})		\Comment{Handles anpassen}
	\State \quad \Method{updateFields}(\Var{source})		\Comment{Übrige Objekte kopieren}
	\State \quad \Method{clear}(\Var{newAdr})
	\State \quad \Method{free}(\Var{source})		\Comment{Alte Version freigeben}
	\Statex
	\State \MethodHead{updateHandles}(\Var{source},\Var{target}):
	\State \quad \FOREACH $\Var{handle} \in \Handles(\Var{source})$		\Comment{Handles der Generation anpassen}
	\State \quad \quad $\Var{obj} \gets \Var{*forward}(\Var{handle})$
	\State \quad \quad \Method{copy}(\Var{obj}, \Var{pos})
	\State \quad \quad $\Var{*origin}(\Var{handle}) \gets \Method{newHandle}(\Var{target}, \Var{newAdr}(\Var{obj}))$
	\Statex
	\State \MethodHead{updateFields}(\Var{source}):
	\State \quad \FOREACH $j > \Var{generation}(\Var{source})$		\Comment{Referenzen aus jüngeren}
	\State \quad \quad \FOREACH $\Var{field} \in \Fields(\textsc{Gen}_j)$	\CommentCont{Generationen anpassen}
	\State \quad \quad \quad \IF ($\Var{*field} \neq \Null \wedge \Var{**field} \in \Var{source}$)
	\State \quad \quad \quad \quad \Method{copy}(\Var{**field}, \Var{pos})
	\State \quad \quad \quad \quad $\Var{*field} \gets \Var{newAdr}(\Var{**field})$
	\State \quad \WHILE $\Var{toDo} \neq \Null$
	\State \quad \quad $\Var{ref} \gets \Method{remove}(\Var{toDo})$
	\State \quad \quad \FOREACH $\Var{field} \in \Fields(\Var{*ref})$	\Comment{Interne Referenzen anpassen}
	\State \quad \quad \quad \IF ($\Var{*field} \neq \Null \wedge \Var{**field} \in \Var{source}$)
	\State \quad \quad \quad \quad \Method{copy}(\Var{**field}, \Var{pos})
	\State \quad \quad \quad \quad $\Var{*field} \gets \Var{newAdr}(\Var{**field})$
	\Statex
	\State \MethodHead{copy}(\Var{obj}, \Var{pos}):
	\State \quad \IF $\Var{newAdr}(\Var{obj}) = \Null$		\Comment{vgl. \Method{update} in Algorithmus~\ref{algo:copying-gc}}
	\State \quad \quad $\Var{newAdr}(\Var{obj}) \gets \Var{pos}$
	\State \quad \quad \Method{moveObject}(\Var{\&obj}, \Var{pos})
	\State \quad \quad $\Var{pos} \gets \Var{pos} + \Method{sizeOf}(\Var{obj})$
	\State \quad \quad \Method{add}(\Var{toDo}, \Var{newAdr}(\Var{obj}))
\end{algorithmic}
\caption[Generationelle Garbage Collection nach \textsc{Lieberman} und \textsc{Hewitt}]{Generationelle Garbage Collection nach \textsc{Lieberman} und \textsc{Hewitt} (vgl. \cite[S. 421ff]{lieberman1983}).}
\label{algo:lieberman}
\end{algorithm}

Das Kopieren eines Objekts wird stets durch die Prozedur \Method{copy} bewerkstelligt.
Genau wie in der Prozedur \Method{update} im Halbraumverfahren (Algorithmus~\ref{algo:copying-gc}) wird ein Objekt nur dann kopiert, wenn es nicht zuvor bereits kopiert wurde.
Andernfalls genügt es, lediglich die im Feld gespeicherte Adresse anzupassen.
Da kopierte Objekte ebenfalls Referenzen auf Objekte derselben Generation enthalten können, werden sie zu \Var{toDo} hinzugefügt.
Bei der Abarbeitung von \Var{toDo} (Zeile~21 bis 26) werden somit auch Objekte aufgespürt, die nur durch generationeninterne Referenzen erreichbar sind.
Abbildung~\ref{fig:lieberman-example} zeigt die Arbeitsweise am Beispiel der oben dargestellten Objektkonstellation.

\newpage

\begin{figure}[H] \newcommand{\liebermanscale}{0.75} \newcommand{\liebermanspace}{1cm}
	\centering
	\begin{subfigure}[t]{0.45\textwidth}
		\centering
		\includestandalone[scale=\liebermanscale]{img/tikz/ch5-lieberman2}
		\caption{Anlegen einer neuen Version.}
	\end{subfigure}~\hspace{\liebermanspace}~
	\begin{subfigure}[t]{0.45\textwidth}
		\centering
		\includestandalone[scale=\liebermanscale]{img/tikz/ch5-lieberman3}
		\caption{Kopieren der von älteren Generationen referenzierten Objekte und Erzeugung neuer Handles.}
	\end{subfigure}\\[1cm]
	\begin{subfigure}[t]{0.45\textwidth}
		\centering
		\includestandalone[scale=\liebermanscale]{img/tikz/ch5-lieberman4}
		\caption{Kopieren der von jüngeren Generationen referenzierten Objekte.}
	\end{subfigure}~\hspace{\liebermanspace}~
	\begin{subfigure}[t]{0.45\textwidth}
		\centering
		\includestandalone[scale=\liebermanscale]{img/tikz/ch5-lieberman5}
		\caption{Abarbeitung der \Var{toDo}-Menge zur Aktualisierung interner Referenzen.}
	\end{subfigure}\\[1cm]
	\begin{subfigure}[t]{0.45\textwidth}
		\centering
		\includestandalone[scale=\liebermanscale]{img/tikz/ch5-lieberman6}
		\caption{\Var{toDo} ist abgearbeitet. Nicht kopierte Elemente in \Var{source} sind verwaist.}
	\end{subfigure}~\hspace{\liebermanspace}~
	\begin{subfigure}[t]{0.45\textwidth}
		\centering
		\includestandalone[scale=\liebermanscale]{img/tikz/ch5-lieberman7}
		\caption{\Var{source} wird en bloc freigegeben.}
	\end{subfigure}\\[0.5cm]
	\caption[Beispielhafte Ausführung von Algorithmus~\ref{algo:lieberman}]{Beispielhafte Ausführung von Algorithmus~\ref{algo:lieberman} für Generation $k = 1$.}
	\label{fig:lieberman-example}
\end{figure}

\newpage

Die Stärke des Algorithmus besteht nun darin, die Häufigkeit der Garbage Collection von der Generation von Objekten und der Versionszahl abhängig zu machen.
Sehr junge Generationen niedriger Version enthalten nach der Hypothese von \textsc{Deutsch} und \textsc{Bobrow} potenziell viele Objekte, die bereits verwaist sind.
Entsprechend ist es attraktiv, diese zu bereinigen, da sie eine vielversprechende Ausbeute bieten.
Die erwartete Laufzeit des Kollektors hält sich dabei gleichzeitig in Grenzen, da nur wenige Nachfolgegenerationen existieren, die Referenzen auf die zu reinigende Generation besitzen könnten.
Ältere Generationen, die schon häufig bereinigt wurden, enthalten wiederum tendenziell weniger verwaiste Objekte.
Da bei ihrer Bereinigung viele jüngere Generationen betrachtet werden müssen, sollte diese eher selten stattfinden (vgl. \cite[S. 423]{lieberman1983}).

Abgesehen von Generation und Versionszahl gibt es noch weitere Stellschrauben zur Optimierung des Algorithmus.
Eine periodische Erhöhung der Generationenzahl hat den Vorteil, dass Objekte, die fast zeitgleich erzeugt werden, in derselben Generation gespeichert werden, was sich für einige Datenstrukturen positiv auf die räumliche Lokalität auswirken dürfte.
Alternativ kann jedoch auch eine neue Generation erst dann angelegt werden, wenn die aktuelle nicht mehr genügend freien Speicher besitzt, um eine Speicheranforderung zu erfüllen.
Das verringert den möglichen Speicherverschnitt durch nur teilweise belegte Blöcke.
Da neue Objekte stets in der jüngsten Generation angelegt werden, tendieren ältere Generationen dazu, mit jedem Kollektionszyklus zu schrumpfen.
Aus demselben Grund bietet es sich daher an, die Größe jeder neuen Version einer Generation auf die Größe aktuell erhaltenen Objekte zu setzen.
Eine andere Möglichkeit wäre, zusätzliche Konsolidierungsphasen einzubauen, in welchen Generationen verschmolzen werden.
Eine weitere, nicht zu verachtende Maßnahme ist die Parallelisierung des Algorithmus, indem mehrere Generationen gleichzeitig gepflegt werden (vgl. \cite[S. 426]{lieberman1983}.

Zuletzt sollte beachtet werden, dass die eingangs erwähnte Hypothese über die Lebensdauer von Objekten in der Praxis nicht zutreffen muss, wenngleich sie auch in neueren Programmiersprachen empirisch bestätigt wurde (vgl. \cite{jones2008}).
Ebenso mag die Annahme, dass nur wenige Referenzen von älteren auf jüngere Objekte zu Stande kommen und sich die Zahl der benötigten Handles damit in Grenzen hält, auf die von \textsc{Lieberman} und \textsc{Hewitt} betrachteten LISP-Systeme zutreffen, da die Komponenten eines Objekts in den meisten Fällen unveränderlich sind und bereits vor Erzeugung des Objekts existieren (vgl. \cite[S. 422]{lieberman1983}).
In objektorientierten Programmiersprachen etwa sind Attribute von Objekten in der Regel mutabel, weswegen dieser Gedanke nicht notwendigerweise auf diese Fälle übertragbar ist.
Die Anzahl der Handles liegt dann viel höher, womit die in Abschnitt~\ref{sec:handle} angesprochenen Nachteile relevant werden können.


% % % % % % % % % % % % % % % % % % % % % % % % % % % % % % % % % % 
 % % % % % % % % % % % % % % % % % % % % % % % % % % % % % % % % % % 
% % % % % % % % % % % % % % % % % % % % % % % % % % % % % % % % % % 


\section{Verborgene Referenzzählung}
\label{sec:ulterior}

Eine periodische Erzeugung neuer Heapbschnitte hat zwar den Vorteil, eine feingranularere Abstimmung der Garbage Collection angepasst an die Lebensdauer der Objekte zu ermöglichen, aber auch den Nachteil, zusätzliche Metainformationen über jede Generation verwalten zu müssen (siehe dazu auch die in Abschnitt~\ref{sec:lazy-sweep} genannten Nachteile).
Als Kompromiss kann es zielführend sein, gröber zwischen jüngeren und älteren Objekten zu entscheiden und sich somit auf zwei Altersklassen zu beschränken:
Jüngere Objekte tendieren nicht nur zu einer hohen Verwaisungswahrscheinlichkeit, sondern auch zu einer höheren Veränderungsrate, während ältere Objekte eher seltener manipuliert werden.
Diese Feststellung nutzen \textsc{Blackburn} und \textsc{McKinley} in ihrem Konzept der \textit{verborgenen Referenzzählung} (engl. \textit{ulterior reference counting}), um die Vorteile markierender und referenzzählender Algorithmen zu vereinigen \cite{blackburn2003}.
Der Heap wird dazu aufgeteilt in zwei (nicht notwendigerweise gleich große) Teile \Nursery (dt. etwa \textit{Kinderzimmer}) und \Mature (dt. \textit{erwachsen}, \textit{ausgewachsen}).
Neu erzeugte Objekte werden zunächst in \Nursery angelegt und verbleiben dort bis zum nächsten Kollektionszyklus.
In \Mature befinden sich alle Objekte, die mindestens einen Kollektionszyklus überlebt haben.

\begin{figure}[h]
	\centering
	\includestandalone[scale=0.9]{img/tikz/ch5-ulterior-principle}
	\caption[Prinzip der verborgenen Referenzzählung]{Prinzip der verborgenen Referenzzählung. Das Schema zeigt, wie Referenzen zwischen den einzelnen Heapbereichen durch die Referenzzählung behandelt werden (vgl. \cite[S. 346]{blackburn2003}).}
	\label{fig:ulterior-principle}
\end{figure}

Ob und in welcher Form die Referenzzählung zum Einsatz kommt, richtet sich nun danach, ob sich Quelle oder Ziel in \Roots, \Nursery oder \Mature befinden (siehe Abbildung~\ref{fig:ulterior-principle}).
Bereits in Abschnitt~\ref{sec:naive-rc} haben wir festgestellt, dass Referenzzählungen -- bedingt durch nötige Schreibbarrieren -- die Performance einer Anwendung beeinträchtigt, wenn viele Manipulationen von Referenzen stattfinden.
Da junge Objekte in \Nursery tendenziell häufiger Manipulationen von eingehenden wie ausgehenden Referenzen unterworfen sind, bietet es sich an, diese von der Referenzzählung auszusparen.
Referenzen von \Roots nach \Nursery sowie innerhalb von \Nursery sind von der Schreibbarriere nicht betroffen; Referenzen von \Mature nach \Nursery werden zunächst protokolliert und während des nächsten Kollektionszyklus behandelt.
Objekte in \Mature verändern sich tendenziell seltener, weshalb Markierungs- und Kompaktierungsalgorithmen keine hohe Ausbeute versprechen, aber dennoch aufwendig sind.
Schreibbarriere und Zähleranpassungen sind für Referenzen innerhalb \Mature daher tolerabel.
Manipulationen, die Referenzen von \Roots bzw. \Nursery nach \Mature betreffen, werden hingegen verzögert behandelt und ihre Auswirklungen auf Referenzzähler erst während einer Kollektionsphase beachtet.\footnote{Zur Erinnerung: Ein Referenzzählerwert $\Var{rc}(\Var{a}) = 0$ bedeutet somit nicht, dass \Var{a} nicht erreichbar ist, sondern dass keine Referenzen auf \Var{a} von Objekten aus \Mature existieren. Jedoch können Referenzen auf \Var{a} von Objekten aus \Roots oder \Nursery existieren (siehe Abschnitt~\ref{sec:rc-optimizing}).}
Algorithmus~\ref{algo:ulterior-barrier} zeigt die entsprechende Schreibbarriere.

\begin{algorithm}[h]
\begin{algorithmic}[1]
	\State \MethodHead{writeRef}(\Var{obj}, \Var{i}, \Var{ref}):
	\State \quad \IF $\Var{obj} \in \Mature$
	\State \quad \quad \Atomic
	\State \quad \quad \quad \IF ($\Var{*obj[i]} \in \Nursery \wedge \Var{*ref} \notin \Nursery$)	
	\State \quad \quad \quad \quad \Method{removeLog}(\Var{\&obj[i]})	\Comment{Falls \Mature $\rightarrow$ \Nursery entfernt wird}
	\State \quad \quad \quad \ELSE \IF ($\Var{*obj[i]} \notin \Nursery \wedge \Var{*ref} \in \Nursery$)
	\State \quad \quad \quad \quad \Method{addLog}(\Var{\&obj[i]})	\Comment{Falls \Mature $\rightarrow$ \Nursery erzeugt wird}
	\State \quad \quad \quad \IF $\Var{*ref} \in \Mature$		\Comment{Referenzen \Mature $\rightarrow$ \Mature behandeln}
	\State \quad \quad \quad \quad \Method{incRefCount}(\Var{*ref})
	\State \quad \quad \quad \IF $\Var{*obj[i]} \in \Mature$
	\State \quad \quad \quad \quad \Method{decRefCount}(\Var{*obj[i]})
	\State \quad $\Var{obj[i]} \gets \Var{ref}$
\end{algorithmic}
\caption[Schreibbarriere der verborgenen Referenzzählung]{Schreibbarriere der verborgenen Referenzzählung (vgl. \cite[S. 346ff]{blackburn2003}).}
\label{algo:ulterior-barrier}
\end{algorithm}

Kommt es nun zu einem Garbage-Collection-Zyklus, etwa weil der freie Speicher im \Nursery-Bereich zu Neige geht, werden alle erreichbare Objekte aus \Nursery nach \Mature kopiert.
Da die Objekte in \Nursery von der Referenzzählung ausgeschlossen sind, wird dies analog zum Halbraumverfahren aus Abschnitt~\ref{sec:copying} bewerkstelligt.
Zunächst werden daher alle Felder von Basisobjekten untersucht (Zeile~2 bis~5 aus Algorithmus~\ref{algo:ulterior-rc}).
Enthält ein Feld eine Referenz auf ein junges Objekt, wird es an die Prozedur \Method{tenure} übergeben.
Diese kopiert ein zuvor noch nicht entdecktes Objekt nach \Mature, fügt es zur Menge \Var{toDo} hinzu und aktualisiert die Speicheradresse im übergebenen Feld (vgl. Prozedur \Method{update} in Algorithmus~\ref{algo:copying-gc}).
Im Anschluss wird -- unabhängig davon, ob das Objekt kopiert wurde -- der Referenzzähler des Objekts inkrementiert, da es sich nun in jedem Fall in \Mature befindet und mittels Referenzzählung verwaltet wird.
Zudem wird hierdurch der Referenzzähler um die Anzahl an Referenzen aus \Roots korrigiert (siehe Algorithmus~\ref{algo:deferred-rc}).

Weiter werden alle Referenzen von älteren auf jüngere Objekte behandelt (Zeile~6 bis~9).
Felder von Objekten in \Mature, die eine Referenz auf Objekte aus \Nursery enthalten, wurden zuvor von der Schreibbarriere zur Menge \Var{log} hinzugefügt.
Somit genügt eine Iteration über diese Menge, um die entsprechenden Objekte zu kopieren.
Auch hier werden wieder alle betroffenen Referenzzähler korrigiert, da Referenzen von \Mature nach \Nursery nun Referenzen innerhalb \Mature sind.
Anschließend erfolgt eine Abarbeitung der \Var{toDo}-Menge, die alle Objekte enthält, die in diesem Kollektionszyklus nach \Mature kopiert wurden (Zeile~10 bis~15).
Da diese Referenzen auf bislang unentdeckte Objekte in \Nursery besitzen können, werden ihre Felder entsprechend untersucht, die Ziele gegebenenfalls kopiert und die Referenzzähler korrigiert.

\begin{algorithm}[h]
\begin{algorithmic}[1]
	\State \Atomic \MethodHead{collectGarbage}():
	\State \quad \FOREACH $\Var{field} \in \Fields(\Roots)$
	\State \quad \quad \IF $\Var{*field} \in \Nursery$		\Comment{Referenzen \Roots $\rightarrow$ \Nursery behandeln}
	\State \quad \quad \quad \Method{tenure}(\Var{field})	\Comment{Ziel nach \Mature kopieren und Feld anpassen}
	\State \quad \quad \Method{incRefCount}(\Var{**field})	\Comment{vgl. Algorithmus~\ref{algo:deferred-rc}}
	\State \quad \FOREACH $\Var{field} \in \Var{log}$	\Comment{Referenzen \Mature $\rightarrow$ \Nursery behandeln}
	\State \quad \quad \Method{tenure}(\Var{field})
	\State \quad \quad \Method{incRefCount}(\Var{**field})
	\State \quad \quad \Method{removeLog}(\Var{field})
	\State \quad \WHILE $\Var{toDo} \neq \emptyset$
	\State \quad \quad $\Var{ref} \gets \Method{remove}(\Var{toDo})$
	\State \quad \quad \FOREACH $\Var{field} \in \Fields(\Var{*ref})$
	\State \quad \quad \quad \IF $\Var{*field} \in \Nursery$	\Comment{Referenzen \Nursery $\rightarrow$ \Nursery behandeln}
	\State \quad \quad \quad \quad \Method{tenure}(\Var{field})
	\State \quad \quad \quad \Method{incRefCount}(\Var{**field})
	\State \quad \Method{free}(\Var{ZCT})	\Comment{Objekte mit $\Var{rc} = 0$ entfernen}
	\State \quad \Method{free}(\Nursery)	\Comment{Verbleibende junge Objekte entfernen}
	\State \quad \FOREACH $\Var{ref} \in \Pointers(\Roots)$	\Comment{RC-Anpassungen rückgängig machen}
	\State \quad \quad \Method{decRefCount}(\Var{*ref})	\Comment{vgl. Algorithmus~\ref{algo:deferred-rc}}
\end{algorithmic}
\caption[Verborgene Referenzzählung nach \textsc{Blackburn} und \textsc{McKinley}]{Verborgene Referenzzählung nach \textsc{Blackburn} und \textsc{McKinley} (vgl. \cite[S. 346ff]{blackburn2003}).}
\label{algo:ulterior-rc}
\end{algorithm}

Alle Objekte, die sich abschließend in \Mature befinden und deren Referenzzähler $0$ beträgt, werden weder von Basisobjekten, noch von anderen Objekten aus \Mature referenziert und sind unerreichbar.
Somit können sie durch Bereinigen der \textit{zero count table} \Var{ZCT} freigegeben werden (Zeile~16, siehe auch Algorithmus~\ref{algo:deferred-rc}).
Objekte, die sich in \Nursery befinden und erreichbar sind, wurden nach \Mature kopiert.
Folglich kann der gesamte Bereich freigegeben werden (Zeile~17).
Zuletzt müssen die Korrekturen der Referenzzähler rückgängig gemacht werden.
Dazu genügt es, nochmals alle von \Roots ausgehenden Referenzen zu betrachten, da diese nun lediglich auf Objekte aus \Mature verweisen.

Ein ausführliches Beispiel zum Ablauf des Algorithmus, das sämtliche Arten von Referenzen zwischen den einzelnen Heapbereichen berücksichtigt, befindet sich im Anhang dieser Arbeit (siehe Anhang~\ref{cha:ulterior-example}), weswegen wir an dieser Stelle abschließend auf die Vor- und Nachteile eingehen.
Ist die Aufteilung des Heaps in \Nursery und \Mature geschickt gewählt, vermag der vorgestellte Algorithmus die Schwächen der in den vorigen Kapiteln vorgestellten Algorithmen zu reduzieren.
Wie bereits erläutert, wird durch die Aussparung jüngerer Objekte der Performanceverlust durch Zählermanipulationen teilweise vermieden.
Die Integration der verzögerten Referenzzählung nach \textsc{Deutsch} und \textsc{Bobrow} verhindert zudem Löschkaskaden und die unmittelbare Nachverfolgung von Referenzmanipulationen in Basisobjekten, etwa bei lokalen Variablen einer Prozedur.
Die Verwaltung älterer Objekte mithilfe der Referenzzählung erspart wiederum die Kosten, die Markierungs- und Kompaktierungsansätze bei Betrachtung des gesamten Heaps mit sich bringen würden.
Hier wirkt sich auch die Protokollierung derjenigen Felder in \Mature aus, die Referenzen nach \Nursery enthalten, da keine gesamte Traversierung des Heaps notwendig ist, um Referenzen auf kopierte Objekte anzupassen.
Die Unterbrechungen im Programmablauf werden damit deutlich verringert.
Beachtet werden muss jedoch die Schwierigkeit, das richtige Verhältnis von \Nursery und \Mature zu finden:
Ist \Nursery zu klein angelegt, verursacht dies häufigere -- wenn auch kürzere -- Kollektionszyklen.
Zudem besteht die Gefahr, dass Objekte frühzeitig nach \Mature verschoben werden, obwohl sie in naher Zukunft noch häufig manipuliert werden.
Ist \Nursery jedoch zu groß, verlängern sich potenziell die Kollektionsphasen, da mehr Referenzen verfolgt werden müssen.
Zwei weitere Probleme bereffen vor allem die \Mature-Region:
Zum einen vermag der Speicher in diesem Bereich nach und nach zu fragmentieren, was jedoch durch geschicktes Kopieren an geeignete Positionen in Grenzen gehalten werden kann.
Zum anderen besitzt der vorgestellte Algorithmus keine Möglichkeit, verwaiste zyklische Referenzen älterer Objekte zu erkennen.
\textsc{Blackburn} und \textsc{McKinley} empfehlen daher, beim Erreichen eines gewissen Füllstandes des Heaps zusätzliche Erkennungen verwaister Zyklen einzustreuen (vgl. \cite[S. 349]{blackburn2003}).

