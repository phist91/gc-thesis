%!TEX root = thesis.tex
% Author: Phil Steinhorst, p.st@wwu.de

% **************************************************
% Document Class Definition
% **************************************************
\documentclass[%
	paper=A4,					% paper size --> A4 is default in Germany
	twoside=true,				% onesite or twoside printing
	openright,					% doublepage cleaning ends up right side
	parskip=full,				% spacing value / method for paragraphs
	chapterprefix=true,			% prefix for chapter marks
	11pt,						% font size
	headings=normal,			% size of headings
	bibliography=totoc,			% include bib in toc
	listof=totoc,				% include listof entries in toc
	titlepage=on,				% own page for each title page
	captions=tableabove,		% display table captions above the float env
	draft=false,				% value for draft version
]{scrreprt}%

% **************************************************
% Debug LaTeX Information
% **************************************************
%\listfiles

% **************************************************
% Information and Commands for Reuse
% **************************************************
\newcommand{\thesisTitle}{Titel der Masterarbeit}
\newcommand{\thesisName}{Phil Steinhorst}
\newcommand{\thesisSubject}{Masterarbeit}
\newcommand{\thesisDate}{\today}
\newcommand{\thesisVersion}{My First Draft}
\newcommand{\thesisAuthorStreet}{Dürerstraße 1}
\newcommand{\thesisAuthorCity}{Münster}
\newcommand{\thesisAuthorPostalCode}{48147}
\newcommand{\thesisAuthorMail}{p.st@wwu.de}
\newcommand{\thesisDegree}{Master of Education}
\newcommand{\thesisAuthorFirstSubject}{Mathematik}
\newcommand{\thesisAuthorSecondSubject}{Informatik}
\newcommand{\thesisAuthorId}{382 837}

\newcommand{\thesisFirstReviewer}{Prof. Dr. Jan Vahrenhold}
\newcommand{\thesisFirstReviewerUniversity}{\protect{Clean Thesis Style University}}
\newcommand{\thesisFirstReviewerDepartment}{Department of Clean Thesis Style}

\newcommand{\thesisSecondReviewer}{Prof. Dr. Markus Müller-Olm}
\newcommand{\thesisSecondReviewerUniversity}{\protect{Clean Thesis Style University}}
\newcommand{\thesisSecondReviewerDepartment}{Department of Clean Thesis Style}

\newcommand{\thesisFirstSupervisor}{Jane Doe}
\newcommand{\thesisSecondSupervisor}{John Smith}

\newcommand{\thesisUniversity}{\protect{Westfälische Wilhelms-Universität Münster}}
\newcommand{\thesisUniversityDepartment}{Fachbereich 10 -- Mathematik und Informatik}
\newcommand{\thesisUniversityInstitute}{Institut für Informatik}
\newcommand{\thesisUniversityGroup}{Clean Thesis Group (CTG)}
\newcommand{\thesisUniversityCity}{Münster}
\newcommand{\thesisUniversityStreetAddress}{Einsteinstraße 62}
\newcommand{\thesisUniversityPostalCode}{48149}

% **************************************************
% Load and Configure Packages
% **************************************************
\usepackage[utf8]{inputenc}		% defines file's character encoding
\usepackage[ngerman]{babel} % babel system, adjust the language of the content
\usepackage[					% clean thesis style
	sansserif=false,%
	hangfigurecaption=false,%
	hangsection=true,%
	hangsubsection=true,%
	figuresep=colon,%
	colorize=reduced,%
	colortheme=bluemagenta,%
	bibsys=biber,%
	bibfile=bib-refs,%
	bibstyle=alphabetic,%
]{cleanthesis}

\usepackage{nimbusmononarrow}

\hypersetup{					% setup the hyperref-package options
	pdftitle={\thesisTitle},	% 	- title (PDF meta)
	pdfsubject={\thesisSubject},% 	- subject (PDF meta)
	pdfauthor={\thesisName},	% 	- author (PDF meta)
	plainpages=false,			% 	-
	colorlinks=false,			% 	- colorize links?
	pdfborder={0 0 0},			% 	-
	breaklinks=true,			% 	- allow line break inside links
	bookmarksnumbered=true,		%
	bookmarksopen=true			%
}

\usepackage{mathtools}
\usepackage{amssymb}
\usepackage[textsize=small,color=Red1!80!OrangeRed1!80]{todonotes}

% TikZ
% ===========================================================
	\usepackage{tikz}
	\usepackage{tikz-cd}					% kommutative Diagramme
	\usetikzlibrary{arrows.meta}			% mehr Pfeile!
	\usetikzlibrary{shadows}
	\usetikzlibrary{calc}
	\usetikzlibrary{positioning}
	\tikzset{>=Latex}						% Standard-Pfeilspitze
% ===========================================================

% Listen und Tabellen
% ===========================================================
	\usepackage{multicol}
	\usepackage{multirow}
	%\setlist{itemsep=0pt}
	\setlist[enumerate]{font=\sffamily\bfseries}
	%\setlist[itemize]{label=$\triangleright$,font=\bfseries}
	\usepackage{tabularx}
% ===========================================================

% thmtools
% ===========================================================
%\usepackage{ntheorem}
%\declaretheoremstyle[%
%	headfont=\bfseries,
%	notefont=\normalfont,
%	bodyfont=\normalfont,
%	headformat=\NAME \ \NUMBER \NOTE,
%	headpunct={\\},
%	postheadspace=1ex,
%	spaceabove=15pt,spacebelow=10pt]%
%{mainstyle}

\usepackage{amsthm}
\newtheoremstyle{satzstyle}%
	{\parskip}% space above
	{0pt}% space below
	{\normalfont}% body font
	{}% indent amount
	{\bfseries}% theorem head font
	{:}% theorem head punctuation
	{\newline}% space after theorem head
	{}% theorem head spec
	
\newtheoremstyle{beweisstyle}%
	{0pt}% space above
	{0pt}% space below
	{\normalfont}% body font
	{}% indent amount
	{\itshape}% theorem head font
	{:}% theorem head punctuation
	{\newline}% space after theorem head
	{}% theorem head spec

\theoremstyle{satzstyle}
\newtheorem{satz}{Satz}[chapter]
\newtheorem{defn}{Definition}[chapter]

%\theoremstyle{beweisstyle}

\makeatletter
\renewenvironment{proof}[1][\proofname]{\par
\pushQED{\qed}%
\normalfont \topsep6\p@\@plus6\p@\relax
\trivlist
\item\relax
{\itshape
#1\@addpunct{:}}\hspace\labelsep\ignorespaces
}{%
\popQED\endtrivlist\@endpefalse
}
\makeatother
% ===========================================================

% minted
% ===========================================================
\usepackage{minted}
\setminted{%
	style=bw,
	fontsize=\normalsize,
	breaklines,
	breakanywhere=false,
	breakbytoken=false,
	breakbytokenanywhere=false,
	breakafter={.,},
	autogobble,
	numbersep=3mm,
	tabsize=4,
	frame=lines
}
\setmintedinline{%
	fontsize=\normalsize,
	numbers=none,
	numbersep=12pt,
	tabsize=4,
}

% ===========================================================

% algorithmics
% ===========================================================
\usepackage[chapter]{algorithm}
\floatname{algorithm}{Algorithmus}
\usepackage{algpseudocode}

\newcommand{\code}[1]{\texttt{#1}}

\newcommand{\Atomic}{\textbf{atomic}\xspace}
\newcommand{\Deref}{\texttt{*}}
\newcommand{\ELSE}{\textbf{else}\xspace}
\newcommand{\FOR}{\textbf{for}\xspace}
\newcommand{\FOREACH}{\textbf{for each}\xspace}
\newcommand{\IF}{\textbf{if}\xspace}
\newcommand{\Method}[1]{\textit{#1}}
\newcommand{\MethodHead}[1]{\textbf{\textit{#1}}}
\newcommand{\Null}{\texttt{null}\xspace}
\newcommand{\Var}[1]{\texttt{#1}}
\newcommand{\WHILE}{\textbf{while}\xspace}

\newcommand{\Pointers}{\textsc{Pointers}\xspace}
\newcommand{\Roots}{\textsc{Roots}\xspace}
\newcommand{\Reach}{\ensuremath{\mathcal{R}}}
% ===========================================================
\input{MathCmds.tex}
\raggedbottom