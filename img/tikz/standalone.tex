%!TEX TS-program = pdflatex
% Author: Phil Steinhorst, p.st@wwu.de

\documentclass[11pt]{standalone}
\usepackage[utf8]{inputenc}
\usepackage[ngerman]{babel}


\usepackage[usenames, table, x11names, final]{xcolor}
\definecolor{ctcolormain}{cmyk}{1, .50, .10, .01}%
\definecolor{ctcoloraccessory}{cmyk}{.18, .98, .18, 0}%
\definecolor{ctcolorblack}{gray}{0}
\definecolor{ctcolorgray}{gray}{.5}
\definecolor{ctcolorgraylight}{gray}{.8}

\usepackage[T1]{fontenc}
\usepackage{lmodern}	% font set: Latin Modern
\usepackage{charter}	% font set: Charter

\usepackage{microtype}			% typographic tuning
\usepackage{setspace}			% for line spacing, e.g. \onehalfspacing
\usepackage[]{graphicx}			% graphics import features
\usepackage[shortlabels]{enumitem}			% for simple list modifications
\usepackage{textcomp}			% different symbols
\usepackage{hyperref} 			% interactive link inside the document

\usepackage[					% advanced quotes
	strict=true,					% 	- warning are errors now
	style=german					% 	- german quotes
]{csquotes}

\setstretch{1.2}					% value for line spacing, use \setstretch{} or \singlespacing or \onehalfspacing or \doublespacing
\setlength{\parindent}{0em}

\usepackage{nimbusmononarrow}
\usepackage{mathtools}
\usepackage{amssymb}
\usepackage{xspace}

% TikZ
% ===========================================================
	\usepackage{tikz}
	\usepackage{tikz-cd}					% kommutative Diagramme
	\usetikzlibrary{arrows.meta}			% mehr Pfeile!
	\usetikzlibrary{shadows}
	\usetikzlibrary{calc}
	\usetikzlibrary{positioning}
	\tikzset{>=Latex}						% Standard-Pfeilspitze
	
	\newcommand{\lispbox}[2]{
		\draw[very thick] (#1+1,#2+1) -- (#1+1,#2) -- (#1,#2) -- (#1,#2+1) -- (#1+2,#2+1) -- (#1+2,#2) -- (#1+1,#2);
		\draw[fill=black] (#1+0.5,#2+0.5) circle (0.2);
		\draw[fill=black] (#1+1.5,#2+0.5) circle (0.2);
	}
% ===========================================================

\usepackage{pgfplots}
\usepackage{wasysym}
\usepackage{latexsym} 						% zusätzliche Symbole
\usepackage{stmaryrd} 						% für Blitz
\usepackage{nicefrac} 						% schräge Brüche
\usepackage{cancel} 						% Befehle zum Durchstreichen
\usepackage{extarrows}						% mehr Pfeile
\usepackage{mathdots}
\usepackage{stackrel}
\usepackage{marvosym}

%!TEX root = thesis.tex
% Abkürzungen
% ===========================================================
	\newcommand{\BB}{\mathbb{B}}
	\newcommand{\CC}{\mathbb{C}}
	\newcommand{\EE}{\mathbb{E}}
	\newcommand{\FF}{\mathbb{F}}
	\newcommand{\HH}{\mathcal{H}}
	\newcommand{\KK}{\mathbb{K}}
	\newcommand{\LL}{\mathbb{L}}
	\newcommand{\NN}{\mathbb{N}}
	\newcommand{\QQ}{\mathbb{Q}}
	\newcommand{\RR}{\mathbb{R}}
	\newcommand{\ZZ}{\mathbb{Z}}
	\newcommand{\oh}{\mathcal{O}}				% Landau-O
	\newcommand{\ind}{1\hspace{-0,8ex}1} 		% Indikatorfunktion (Doppeleins)
	\newcommand{\bewrueck}{\enquote{$\Leftarrow$}:} 	% Beweis Rückrichtung
	\newcommand{\bewhin}{\enquote{$\Rightarrow$}:}		% Beweis Hinrichtung
	\newcommand{\ol}[1]{\overline{#1}}
	\newcommand{\wt}[1]{\widetilde{#1}}
	\newcommand{\wh}[1]{\widehat{#1}}
% ===========================================================

% Operatoren
% ===========================================================
	\DeclareMathOperator{\id}{id} 				% Identität
	\DeclareMathOperator{\im}{im} 				% image
	\DeclareMathOperator{\pot}{\mathcal{P}}		% Potenzmenge
	\DeclareMathOperator{\sgn}{sgn} 			% Signum
	\DeclareMathOperator{\Sym}{Sym} 			% Symmetrische Gruppe
% ===========================================================

% Klammerungen und ähnliches
% ===========================================================
	\DeclarePairedDelimiter{\absolut}{\lvert}{\rvert}		% Betrag
	\DeclarePairedDelimiter{\ceiling}{\lceil}{\rceil}		% aufrunden
	\DeclarePairedDelimiter{\Floor}{\lfloor}{\rfloor}		% aufrunden
	\DeclarePairedDelimiter{\Norm}{\lVert}{\rVert}			% Norm
	\DeclarePairedDelimiter{\sprod}{\langle}{\rangle}		% spitze Klammern
	\DeclarePairedDelimiter{\enbrace}{(}{)}					% runde Klammern
	\DeclarePairedDelimiter{\benbrace}{\lbrack}{\rbrack}	% eckige Klammern
	\DeclarePairedDelimiter{\penbrace}{\{}{\}}				% geschweifte Klammern
	\newcommand{\Underbrace}[2]{{\underbrace{#1}_{#2}}} 	% bessere Unterklammerungen
	% Kurzschreibweisen für Faule und Code-Vervollständigung
	\newcommand{\abs}[1]{\absolut*{#1}}
	\newcommand{\ceil}[1]{\ceiling*{#1}}
	\newcommand{\flo}[1]{\Floor*{#1}}
	\newcommand{\no}[1]{\Norm*{#1}}
	\newcommand{\sk}[1]{\sprod*{#1}}
	\newcommand{\enb}[1]{\enbrace*{#1}}
	\newcommand{\penb}[1]{\penbrace*{#1}}
	\newcommand{\benb}[1]{\benbrace*{#1}}
	\newcommand{\stack}[2]{\makebox[1cm][c]{$\stackrel{#1}{#2}$}}
% ===========================================================

% Monotypes
% ===========================================================
	\newcommand{\Band}{\mathtt{AND}}
	\newcommand{\Bor}{\mathtt{OR}}
	\newcommand{\zero}{\mathtt{0}}
	\newcommand{\one}{\mathtt{1}}
	\newcommand{\Bnot}{\mathtt{NOT}}
	\newcommand{\Bnand}{\mathtt{NAND}}
	\newcommand{\Bnor}{\mathtt{NOR}}
	\newcommand{\Bxor}{\mathtt{XOR}}
	\DeclareMathOperator{\DNF}{DNF}
	\DeclareMathOperator{\KNF}{KNF}
	
	\newcommand{\NUM}{\ensuremath{\mathtt{NUM}}}
	\newcommand{\OP}{\ensuremath{\mathtt{OP}}}
	\newcommand{\EXP}{\ensuremath{\mathtt{EXP}}}
	\newcommand{\scheme}{R\textsuperscript{5}RS\xspace}
	\newcommand{\slot}[1]{{\small\ensuremath{\sprod{\mathit{#1}}}}}
	\newcommand{\fslot}[1]{{\footnotesize\ensuremath{\sprod{\mathit{#1}}}}}
	
	%%%%%%%%%%%%%%%%%%%%%%%%%%%%%%%%%%%%%%%%%%%%%%%%%%%%%%%%%%%%%%%%%%%%%%%%%%
	
	\newcommand{\code}[1]{\texttt{#1}}
	
	\newcommand{\Atomic}{\textbf{atomic}\xspace}
	\newcommand{\Deref}{\texttt{*}}
	\newcommand{\Do}{\textbf{do}\xspace}
	\newcommand{\ELSE}{\textbf{else}\xspace}
	\newcommand{\FOR}{\textbf{for}\xspace}
	\newcommand{\FOREACH}{\textbf{for each}\xspace}
	\newcommand{\Global}{\item[\textbf{global:}]}
	\newcommand{\Input}{\item[\textbf{Input:}]}
	\newcommand{\IF}{\textbf{if}\xspace}
	\newcommand{\IFN}{\textbf{if not}\xspace}
	\newcommand{\Method}[1]{\textit{#1}}
	\newcommand{\MethodHead}[1]{\textbf{\textit{#1}}}
	\newcommand{\NOT}{\textbf{not}\xspace}
	\newcommand{\Null}{\texttt{null}\xspace}
	\newcommand{\Pre}{\item[\textbf{Vorbedingung:}]}
	\newcommand{\Var}[1]{\texttt{#1}}
	\newcommand{\WHILE}{\textbf{while}\xspace}
	
	\newcommand{\Fields}{\textsc{PointerFields}\xspace}
	\newcommand{\Pointers}{\textsc{Pointers}\xspace}
	\newcommand{\Roots}{\textsc{Roots}\xspace}
	\newcommand{\Blocks}{\textsc{Blocks}\xspace}
	\newcommand{\Reach}{\ensuremath{\mathcal{R}}}
	\newcommand{\ObjectSet}{\mathbb{O}}
	\newcommand{\HeapSize}{\mathbb{H}}
	\newcommand{\Handles}{\textsc{Handles}\xspace}
	
	\newcommand{\transreach}{\overset{*}{\rightarrow}}
	\newcommand{\handle}[1]{\ensuremath{\mathtt{\tilde{#1}}}}