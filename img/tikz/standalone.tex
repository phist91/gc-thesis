%!TEX TS-program = pdflatex
% Author: Phil Steinhorst, p.st@wwu.de
\documentclass[11pt]{standalone}
\usepackage[utf8]{inputenc}
\usepackage[ngerman]{babel}

\usepackage[usenames, table, x11names, final]{xcolor}
\definecolor{ctcolormain}{cmyk}{1, .50, .10, .01}%
\definecolor{ctcoloraccessory}{cmyk}{.18, .98, .18, 0}%
\definecolor{ctcolorblack}{gray}{0}
\definecolor{ctcolorgray}{gray}{.5}
\definecolor{ctcolorgraylight}{gray}{.8}

\usepackage[T1]{fontenc}
\usepackage{lmodern}	% font set: Latin Modern
\usepackage{charter}	% font set: Charter

\usepackage{microtype}			% typographic tuning
\usepackage{setspace}			% for line spacing, e.g. \onehalfspacing
\usepackage[]{graphicx}			% graphics import features
\usepackage[shortlabels]{enumitem}			% for simple list modifications
\usepackage{textcomp}			% different symbols
\usepackage{hyperref} 			% interactive link inside the document

\usepackage[					% advanced quotes
	strict=true,					% 	- warning are errors now
	style=german					% 	- german quotes
]{csquotes}

\setstretch{1.2}					% value for line spacing, use \setstretch{} or \singlespacing or \onehalfspacing or \doublespacing
\setlength{\parindent}{0em}

\usepackage{nimbusmononarrow}
\usepackage{mathtools}
\usepackage{amssymb}
\usepackage{xspace}

% TikZ
% ===========================================================
	\usepackage{tikz}
	\usepackage{tikz-cd}					% kommutative Diagramme
	\usetikzlibrary{arrows.meta}			% mehr Pfeile!
	\usetikzlibrary{shadows}
	\usetikzlibrary{calc}
	\usetikzlibrary{positioning}
	\tikzset{>=Latex}						% Standard-Pfeilspitze
	
	\newcommand{\lispbox}[2]{
		\draw[very thick] (#1+1,#2+1) -- (#1+1,#2) -- (#1,#2) -- (#1,#2+1) -- (#1+2,#2+1) -- (#1+2,#2) -- (#1+1,#2);
		\draw[fill=black] (#1+0.5,#2+0.5) circle (0.2);
		\draw[fill=black] (#1+1.5,#2+0.5) circle (0.2);
	}
% ===========================================================

\usepackage{pgfplots}
\usepackage{wasysym}
\usepackage{latexsym} 						% zusätzliche Symbole
\usepackage{stmaryrd} 						% für Blitz
\usepackage{nicefrac} 						% schräge Brüche
\usepackage{cancel} 						% Befehle zum Durchstreichen
\usepackage{extarrows}						% mehr Pfeile
\usepackage{mathdots}
\usepackage{stackrel}
\usepackage{marvosym}

\input{../../MathCmds.tex}